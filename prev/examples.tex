\documentclass{article}
\usepackage{amsmath, amsthm, amssymb, amsfonts}
\usepackage{tikz}
\usepackage{filecontents}
\usepackage{listings}
\usepackage[margin=1in]{geometry}


\lstdefinestyle{base}{
  emptylines=1,
  breaklines=true,
  basicstyle=\ttfamily\color{black},
  basicstyle=\footnotesize\ttfamily,
  moredelim=**[is][\color{red}]{@}{@},
}

\usetikzlibrary{arrows}
\begin{document}
\paragraph{Some examples of global threshold utilization:} \hspace{0pt} \\

\begin{filecontents*}{temp.tikz}
\begin{tikzpicture}[->,>=stealth',shorten >=1pt,auto,node distance=5cm,
                    thick,main node/.style={circle,draw,font=\sffamily\small}, align=center, text width=0.5cm]

  \node[main node] (1) {$q_1$};
  \node[main node, node distance=3cm] (2) [right of=1] {$q_2$};
  \node[main node, node distance=6cm] (3) [left of=1] {$q_0$};
  
  \node[align=left,font=\sffamily\small] (C) at (-6.2,1.2) {\text{raise $e_0$, $t_0$}};
  \node[align=left,font=\sffamily\small] (C) at (-6.2,-1.2) {\text{wait $e_1$, $t_1$}};
  \node[align=left,font=\sffamily\small] (D) at (1,1) {\text{east , $e_2$, $t_2$ }};
  \node[align=left,font=\sffamily\small] (D) at (-2.3,0) {\text{west, $e_3$, $t_3$ }\\\text{catch $e_4$, $t_4$ }};
  \node[align=left,font=\sffamily\small] (E) at (1,-1) {\text{west , $e_5$, $t_5$}};
  \node[align=left,font=\sffamily\small] (E) at (4.5,0) {\text{east, $e_6$, $t_6$}\\ \text{catch, $e_7$, $t_7$}};
  \path[every node/.style={font=\sffamily\small}]
    (1) edge [bend left] node [above,sloped] {} (2)
    (1) edge [loop left] node [above,sloped] {} (1)
    (2) edge [bend left] node [below,sloped] {} (1)
    (2) edge [loop right] node [above,sloped] {} (2)
    (3) edge [loop above] node [above,sloped] {} (3)
    (3) edge [loop below] node [above,sloped] {} (3)
\end{tikzpicture}
\end{filecontents*}
\begin{figure}[htb]
\centering
\resizebox{13cm}{!}{ \begin{tikzpicture}[>=latex,shorten >=1pt,node distance=3cm,on grid,auto, node/.style={circle,draw,minimum size=25pt}, ]

 \node[state] (B) at (80pt,0pt) {$s_1$};

 \node[state] (A) at (0pt,0pt) {$s_0$};
 \draw[<-] (A) -- node[above left] {} ++(-1,0);
 \draw[->] (B) to[out=20,in=-20,looseness=8] node[right, align=left] {pass, 1} (B);
 \draw[->] (A) to[out=20,in=160,looseness=1] node[above] {snapshot, 0} (B);
 \draw[->] (B) to[out=200,in=-20,looseness=1] node[below] {move, 0} (A);
 \draw[->] (A) to[out=70,in=110,looseness=8] node[above left, align=left] {move, 2 \\ pass, 0} (A);


 \end{tikzpicture}
}
\caption{Automata $A_0$ on the left and $A_1$ on the right}\label{fig:myfigure}
\end{figure}

$\text{\{raise, west, east, wait, catch\}}$ are all possible actions. $e_i$ and $t_i$ are respectively the semiring value and the threshold associated to the corresponding transition.
Each automaton has a global threshold $T_{A_i}$ such that $A_{|T_{A_i}}$ is the restriction of the automaton to all transition involving semiring values below $T_{A_i}$.
In the next figure, some values have been associated with each $e_i$ and $t_i$
\begin{filecontents*}{temp.tikz}
\begin{tikzpicture}[->,>=stealth',shorten >=1pt,auto,node distance=5cm,
                    thick,main node/.style={circle,draw,font=\sffamily\small}, align=center, text width=0.5cm]

  \node[main node] (1) {$q_1$};
  \node[main node, node distance=3cm] (2) [right of=1] {$q_2$};
  \node[main node, node distance=6cm] (3) [left of=1] {$q_0$};
  
  \node[align=left,font=\sffamily\small] (C) at (-6.2,1.2) {\text{raise 1, 3}};
  \node[align=left,font=\sffamily\small] (C) at (-6.2,-1.2) {\text{wait 2, 4}};
  \node[align=left,font=\sffamily\small] (D) at (1,1) {\text{east , 3, 4 }};
  \node[align=left,font=\sffamily\small] (D) at (-2.3,0) {\text{west, 3, 4 }\\\text{catch 6, 7 }};
  \node[align=left,font=\sffamily\small] (E) at (1,-1) {\text{west , 3, 4}};
  \node[align=left,font=\sffamily\small] (E) at (4.5,0) {\text{east, 3, 4}\\ \text{catch, 6, 7}};
  \path[every node/.style={font=\sffamily\small}]
    (1) edge [bend left] node [above,sloped] {} (2)
    (1) edge [loop left] node [above,sloped] {} (1)
    (2) edge [bend left] node [below,sloped] {} (1)
    (2) edge [loop right] node [above,sloped] {} (2)
    (3) edge [loop above] node [above,sloped] {} (3)
    (3) edge [loop below] node [above,sloped] {} (3)
\end{tikzpicture}
\end{filecontents*}
\begin{figure}[htb]
\centering
\resizebox{13cm}{!}{ \begin{tikzpicture}[>=latex,shorten >=1pt,node distance=3cm,on grid,auto, node/.style={circle,draw,minimum size=25pt}, ]

 \node[state] (B) at (80pt,0pt) {$s_1$};

 \node[state] (A) at (0pt,0pt) {$s_0$};
 \draw[<-] (A) -- node[above left] {} ++(-1,0);
 \draw[->] (B) to[out=20,in=-20,looseness=8] node[right, align=left] {pass, 1} (B);
 \draw[->] (A) to[out=20,in=160,looseness=1] node[above] {snapshot, 0} (B);
 \draw[->] (B) to[out=200,in=-20,looseness=1] node[below] {move, 0} (A);
 \draw[->] (A) to[out=70,in=110,looseness=8] node[above left, align=left] {move, 2 \\ pass, 0} (A);


 \end{tikzpicture}
}
\caption{Automata $A_0$ on the left and $A_1$ on the right}\label{fig:myfigure}
\end{figure}

Let $T_{A_0}=3$ and  $T_{A_1}=5$. We assume that raise can only compose with catch, and wait can only compose with west or east. There are now two possibilities to compose $A_1$ and $A_0$. Either :
$$A = (A_0 \otimes A_1)_{|T_{A_0} \otimes T_{A_1}} $$
\begin{filecontents*}{temp.tikz}
\begin{tikzpicture}[->,>=stealth',shorten >=1pt,auto,node distance=5cm,
                    thick,main node/.style={circle,draw,font=\sffamily\small}, align=center, text width=0.5cm]

  \node[main node] (1) {$q_1$};
  \node[main node, node distance=3cm] (2) [right of=1] {$q_2$};

  \node[align=left,font=\sffamily\small] (D) at (1,1) {\text{east-wait , 5, 8 }};
  \node[align=left,font=\sffamily\small] (D) at (-3,0) {\text{west-wait, 5, 8 }\\\text{catch-raise 7, 10 }};
  \node[align=left,font=\sffamily\small] (E) at (1,-1) {\text{west-wait , 5, 8}};
  \node[align=left,font=\sffamily\small] (E) at (4.5,0) {\text{east-wait, 5, 8}\\ \text{catch-raise, 7, 10}};
  \path[every node/.style={font=\sffamily\small}]
    (1) edge [bend left] node [above,sloped] {} (2)
    (1) edge [loop left] node [above,sloped] {} (1)
    (2) edge [bend left] node [below,sloped] {} (1)
    (2) edge [loop right] node [above,sloped] {} (2)
\end{tikzpicture}
\end{filecontents*}
\begin{figure}[htb]
\centering
\resizebox{10cm}{!}{ \begin{tikzpicture}[>=latex,shorten >=1pt,node distance=3cm,on grid,auto, node/.style={circle,draw,minimum size=25pt}, ]

 \node[state] (B) at (80pt,0pt) {$s_1$};

 \node[state] (A) at (0pt,0pt) {$s_0$};
 \draw[<-] (A) -- node[above left] {} ++(-1,0);
 \draw[->] (B) to[out=20,in=-20,looseness=8] node[right, align=left] {pass, 1} (B);
 \draw[->] (A) to[out=20,in=160,looseness=1] node[above] {snapshot, 0} (B);
 \draw[->] (B) to[out=200,in=-20,looseness=1] node[below] {move, 0} (A);
 \draw[->] (A) to[out=70,in=110,looseness=8] node[above left, align=left] {move, 2 \\ pass, 0} (A);


 \end{tikzpicture}
}
\caption{Automata $A=(A_0 \otimes A_1)_{|T_{A_0} \otimes T_{A_1}}$ with $T_{A}=8$}\label{fig:myfigure}
\end{figure}

\newpage
Or :
$$A = {A_0}_{|T_{A_0}} \otimes {A_1}_{|T_{A_1}} $$

\begin{filecontents*}{temp.tikz}
\begin{tikzpicture}[->,>=stealth',shorten >=1pt,auto,node distance=5cm,
                    thick,main node/.style={circle,draw,font=\sffamily\small}, align=center, text width=0.5cm]

  \node[main node] (1) {$q_1$};
  \node[main node, node distance=3cm] (2) [right of=1] {$q_2$};

  \node[align=left,font=\sffamily\small] (D) at (1,1) {\text{east-wait , 5, 8 }};
  \node[align=left,font=\sffamily\small] (D) at (-3,0) {\text{west-wait, 5, 8 }};
  \node[align=left,font=\sffamily\small] (E) at (1,-1) {\text{west-wait , 5, 8}};
  \node[align=left,font=\sffamily\small] (E) at (4.5,0) {\text{east-wait, 5, 8}};
  \path[every node/.style={font=\sffamily\small}]
    (1) edge [bend left] node [above,sloped] {} (2)
    (1) edge [loop left] node [above,sloped] {} (1)
    (2) edge [bend left] node [below,sloped] {} (1)
    (2) edge [loop right] node [above,sloped] {} (2)
\end{tikzpicture}
\end{filecontents*}
\begin{figure}[htb]
\centering
\resizebox{10cm}{!}{ \begin{tikzpicture}[>=latex,shorten >=1pt,node distance=3cm,on grid,auto, node/.style={circle,draw,minimum size=25pt}, ]

 \node[state] (B) at (80pt,0pt) {$s_1$};

 \node[state] (A) at (0pt,0pt) {$s_0$};
 \draw[<-] (A) -- node[above left] {} ++(-1,0);
 \draw[->] (B) to[out=20,in=-20,looseness=8] node[right, align=left] {pass, 1} (B);
 \draw[->] (A) to[out=20,in=160,looseness=1] node[above] {snapshot, 0} (B);
 \draw[->] (B) to[out=200,in=-20,looseness=1] node[below] {move, 0} (A);
 \draw[->] (A) to[out=70,in=110,looseness=8] node[above left, align=left] {move, 2 \\ pass, 0} (A);


 \end{tikzpicture}
}
\caption{Automata $A={A_0}_{|T_{A_0}} \otimes {A_1}_{|T_{A_1}}$ with $T_{A}=8$}\label{fig:myfigure}
\end{figure}

This threshold can be usefull to separate composition of \textit{internal} actions and \textit{external} actions. An action can be defined as \textit{internal} if it involves a single agent (west or east action). An action is called \textit{external} if it needs some other agents to be allowed (catch for instance). In this scope, internal actions have low semiring values, and external actions have high semiring values. By defining a global threshold $T_A$, and a restriction of the Automaton regarding the threshold, we allow only composition of internal action, or internal and external actions.

\end{document}


