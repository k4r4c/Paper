\section{Composition and limits: 1-2 pages}
\paragraph{Synchronous behavior} Components define relations on ports. We identify the name and the relation of a component $C$ and call $\Pi_v C$ the set of streams observe at port $v \in P_{C}$. If $v\notin P_{C}$, we consider all possible stream in the observable set of stream $\Pi_v C$. Given two components $C_1$ and $C_2$, we describe the composite relation $C_1 \bowtie C_2$ as $$C_1 \bowtie C_2 = \Pi_v C_1 \bigcap \Pi_v C_2$$
for all $v \in P_{C_1} \bigcup P_{C_2}$. Intuitively, the composition relation $\bowtie$ joins all tables (as pictured in Table.1) containing equal stream on the shared port.

Composition of component is a new component. From the definition given previously, it is not hard to see that $\bowtie$ is associative, commutative and idempotent.
%A port is shared between two components. Intuitively, we can see a port as a variable in which a writer can put a value, and a reader can get its value. Nodes generalize this notion since they represent the case of multiple writers, and readers. Given two components $A(a_1,...,a_n)$ and $B(b_1,...,b_n)$ where $a_1, ..., a_n \in P_{A}$ and $b_1, ..., b_n \in P_{B}$ denote their boundary ports, the composition $A \bowtie B$ is defined as follow:

% If the port is shared 
%In this case, they must all agree on the data flowing through the port. Ports can be seen as a synchronization point for component in composition.

\paragraph{Partial order} We extend the composition operator acting on components $\bowtie$ to soft constraint automata. Soft constraint automata define a partial order on the set of streams of acceptable behavior. Therefore, besides composing stream of observable behaviors as depicted in previous paragraph, lifting $\bowtie$ to soft constraint automata also induces a partial order on the composed set of observable behavior.
The partial order is induced by the underlying c-semiring of the soft constraint automaton. As we want to compose soft constraint automata with different underlying c-semirings, we need to find the corresponding structure for the composite c-semiring as a c-semiring itself. We state the problem as follows. Given $A_1$ and $A_2$ two soft constraint automata with underlying c-semiring $E_1$ and $E_2$, how to define the partial order on the set of behavior produced by the composite soft constraint automata $A_1 \bowtie A_2$ in terms of the partial order induced by $E_1$ and $E_2$ ?
\paragraph{Composition} 
In current literature, the composition of soft constraint automata is only defined on soft constraint automata involving the same c-semiring. In case of two different c-semirings, the composition requires an homomorphism from the component c-semirings to the product c-semiring. For example, assuming two soft constraint automata $A_1$ and $A_2$ respectively involving two different c-semiring $E_1$ and $E_2$, the composition used in \cite{?} defined two homomorphisms $h_1 : E_1 \rightarrow E$ and $h_2 : E_2 \rightarrow E$ where $E$ is a given product c-semiring. It becomes possible to choose the product c-semiring $E$ such that the composition admits a certain kind of partial order (e.g. join, lexicographical).

However, a more restrictive constraint is that the composition of c-semiring needs to define a new c-semiring so that our composition is well defined. Thus, as explained in \cite{?}, to preserve c-semiring laws after composition, the carrier of the c-semiring should be restricted to its cancellative elements. Note that any c-semiring with idempotent $\times$ are non-cancellative c-semiring, thus can not be used by current composition.


\begin{definition}
	\textbf{Composition} of two soft constraint automata $A_1=(Q_1,\rightarrow_1, C_1 \times E, q_{01})$ and $A_2=(Q_2,\rightarrow_2, C_2\times E, q_{02})$ is defined as the product $ {A_1 \bowtie A_2} = (Q, \rightarrow, C\times E, (q_{01},q_{02}) )$ where : 
	\begin{list}{-}{ }
		\item $Q= Q_1 \times Q_2 $ is a set of state and $(q_{01},q_{02})\in$ Q is the initial state.
		\item Given $\phi_1 \in C_1$ and $\phi_2 \in C_2$, then $\phi_1 \land \phi_2 \in C$.
		\item The transition relation $\rightarrow$ is the smallest relation satisfying
		$$
		\frac{q \xrightarrow{\phi_1,e_1} q' \in \rightarrow _1 \quad , \quad p\xrightarrow{\phi_2,e_2}p' \in \rightarrow_2}{(q,p) \xrightarrow{\phi_1 \land \phi_2,e_1 \times e_2}(q',p') \in \rightarrow}
		$$
		%A transition is denoted by $\left\langle q_i, c, s, p_i \right\rangle \in \rightarrow$
	\end{list}
\end{definition}
\paragraph{Limits}
We argue that it is not necessary to give an interpretation for the homomorphic function to the c-semiring product during composition. Furthermore, it seems that delaying the definition of the product space and the injection function allows the definition of more complex products such as runtime and non-cancellative c-semiring composition.

An example of a non-cancellative c-semiring is the tropical semiring  $(\mathbb{N}\cup\{\infty\},min,+,\infty,0)$. We now look at the composition of a tropical semiring and a weighted semiring, being $(\mathbb{W}\cup{\infty},max,+,\infty,0)$