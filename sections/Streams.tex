\section{Composition and limits: 1-2 pages}
\paragraph{Synchronous behavior} Components define relations on ports. When a port is shared by multiple component, we 
A port is shared between two components. Intuitively, we can see a port as a variable in which a writer can put a value, and a reader can get its value. Nodes generalize this notion since they represent the case of multiple writers, and readers. Given two components $A(a_1,...,a_n)$ and $B(b_1,...,b_n)$ where $a_1, ..., a_n \in P_{A}$ and $b_1, ..., b_n \in P_{B}$ denote their boundary ports, the composition $A \bowtie B$ is defined as follow:

% If the port is shared 
%In this case, they must all agree on the data flowing through the port. Ports can be seen as a synchronization point for component in composition.

\paragraph{Partial order}
Soft constraint automata define a partial order on the set of streams of acceptable behavior. The partial order is induced by the underlying c-semiring of the soft constraint automaton. As we want to compose soft constraint automata with different underlying c-semiring, we need to find the corresponding structure for the composite c-semiring as a c-semiring itself. We state the problem as follows. Given $A_1$ and $A_2$ two soft constraint automata with underlying c-semiring $E_1$ and $E_2$, how to define the partial order on the set of behavior produced by the composite soft constraint automata $A_1 \bowtie A_2$ in terms of the partial order induced by $E_1$ and $E_2$ ?
\paragraph{Composition} 
In current literature, the composition of soft constraint automata is only defined on soft constraint automata involving the same c-semiring. In case of two different c-semiring, the composition requires an homomorphism from the component c-semirings to the product c-semiring. Assuming two soft constraint automata $A_1$ and $A_2$ respectively involving two different c-semiring $E_1$ and $E_2$, the composition used in \cite{?} defined two homomorphisms $h_1 : E_1 \rightarrow E$ and $h_2 : E_2 \rightarrow E$ where $E$ is a product c-semiring. Moreover, we need the composition of c-semiring to define a new c-semiring so that our composition is well defined. Thus, as explained in \cite{?}, to preserve c-semiring properties after composition, the carrier of the c-semiring should be restricted to its cancellative elements. Note that any c-semiring with $\times$ idempotent are non-cancellative (all elements are non-cancellative).
The counter example is illustrated as follow: 
\begin{definition}
	\textbf{Composition} of two soft constraint automata $A_1=(Q_1,\rightarrow_1, C_1 \times E, q_{01})$ and $A_2=(Q_2,\rightarrow_2, C_2\times E, q_{02})$ is defined as the product $ {A_1 \bowtie A_2} = (Q, \rightarrow, C\times E, (q_{01},q_{02}) )$ where : 
	\begin{list}{-}{ }
		\item $Q= Q_1 \times Q_2 $ is a set of state and $(q_{01},q_{02})\in$ Q is the initial state.
		\item Given $\phi_1 \in C_1$ and $\phi_2 \in C_2$, then $\phi_1 \land \phi_2 \in C$.
		\item The transition relation $\rightarrow$ is the smallest relation satisfying
		$$
		\frac{q \xrightarrow{\phi_1,e_1} q' \in \rightarrow _1 \quad , \quad p\xrightarrow{\phi_2,e_2}p' \in \rightarrow_2}{(q,p) \xrightarrow{\phi_1 \land \phi_2,e_1 \times e_2}(q',p') \in \rightarrow}
		$$
		%A transition is denoted by $\left\langle q_i, c, s, p_i \right\rangle \in \rightarrow$
	\end{list}
\end{definition}
\paragraph{Limits}
We argue that it is not necessary to explicitly give a definition of the homomorphic function to the c-semiring product during composition. Furthermore, delaying the definition of the product space and the injection function allows the definition of more complex products such as runtime and non-cancellative c-semiring composition.