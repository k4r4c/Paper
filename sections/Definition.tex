\section{Definition}
%	Definition introduction :
% Let just describe what we need to define our two main points : new composition operator on preferences, and semiring automata.This section present all the main ingredient for the definition of semiring automata in the next section. As we talk about constraint systems, we start with the definition of Constraint Automata (CA), the fundamental brick for our construction. Since constraints are crisps in CA, we then define c-semiring, a mathematical structure in which our preferences live. The last definition refers to Soft Constraint Automata, an extension of Constraint Automata with preferences. Each of the following structure (CA, c-semiring, SCA) is paired with a closed composition operator. Moreover, every definitions are directly motivated by some examples.

\subsection{Constraint Automata}

%	Constraint Automata structure :
%		Introduction 
%			why using constraint automata ?
%			some examples of problems, and some applications.
%		
%
%

%Given the current computer architecture, and computational model, defining a system as a set of variables is adequate. Whether we look down to the transistor of a computer, or into a software program, we always face the same question : how to automatically go from a variable assignment to another variable assignment ?
%Since we want our model to restrict the possible behavior
Constraint Automata (CA) were introduced in \cite{MSA03} and define a compositional framework for the design of concurrent systems. CA are state transition system where transitions are labeled with constraints. We first present the language that defines the constraints, and then define the syntax and semantic of Constraint Automata. 
%From an operational perspective, each constraints represent a boolean condition to go from a source state to a target state. Constraints are first order logical formula in the language of constraint. 

\begin{definition} (Language of constraint) A term is defined as:
	$$ t := \quad v \quad | \quad f(t_1, ..., t_n) \quad | \quad * $$
	A constraint is a formula $\phi$ defined by:
	$$ \phi := \quad \bot \quad |\quad t_1 = t_2 \quad|\quad R(t_1, ... , t_n) \quad|\quad \phi_1 \land \phi_2 \quad| \quad \neg \phi \quad| \quad \exists x \phi $$
	We denote by V$_{\phi}$ the set of all free variables contain in a formula $\phi$, and by D the data domain of the variables.
\end{definition}

%At this point, we are aware that first-order logic is undecidable, meaning a sound, complete and terminating decision algorithm for provability is impossible. However, taking a general definition allows later on a restriction to a certain FOL fragment. 

Terms are either variables, functions or special symbol *. Intuitively, we can see variables' value as data acquisition of sensors. Variables are defined over a data domain D, and the absence of data is represented by the symbol *. A formula $\phi$, represents the constraint over the set of variable $V_{\phi}$. We say that $\phi$ is satisfiable if there exists an assignment $\delta : V_{\phi}\rightarrow D$ such that $\delta \models \phi$. Composition of two constraints $c_1$ and $c_2$ is written as the conjunction $c_1 \land c_2$. If two constraints have disjoint variables, then the composition is satisfiable if and only if the two composite constraints are satisfiable. If the set of variable of two constraints intersect, satisfiability can no more be deduced from the satisfiability of composite constraints.

%\begin{proposition}
	If $V_{c_1} \cap V_{c_2} = \emptyset$, $\delta_1\models c_1$ and $\delta_2 \models c_2$ then $\delta_1,\delta_2 \models c_1 \land c_2$
\end{proposition}


\begin{example}
	We consider three constraints $east := x \not =* \land y=* $, $west := y \not = * \land x=* $ and $charge := c \not = * $, where x, y and c are three variables. We can imagine that the name of the constraints $east$, $west$ and $charge$ reflects to the actual behavior of our system : having the constraint $east$ true results in the $east$ behavior (respectively with $west$ and $charge$). Since $V_{east} \cap V_{charge} = \emptyset$, if $east$ is satisfiable and $charge$ is satisfiable, then $east \land charge$ is also satisfiable. Looking at the composition of $east$ and $west$, the intersection $V_{east} \cap V_{west} \not= \emptyset$ and, in this case, the composed constraint is not satisfiable. 
\end{example}



\begin{definition}
	A Constraint Automaton is a tuple $\left\langle Q, \rightarrow, C, q_{0}, \right\rangle$ where: 
	\begin{list}{-}{ }
		\item Q is a set of state and $q_0\in$ Q is the initial state.
%		\item C is a set of constraints, where a constraint is a predicate $c: (V_c \rightarrow D) \rightarrow \{\top,\bot\}$.
		\item C is a set of constraints.
		\item ${\rightarrow} \subseteq Q \times C \times Q$ is a finite relation called transition relation.
	\end{list}
		The tuple $(q,c,q') \in \rightarrow$ represents a transition from state q to state q' subject to the constraint c. We write $q \xrightarrow{c} q'$ instead of $(q,c,q') \in \rightarrow$. If c $\equiv \bot$, we can drop $q \xrightarrow{c} q'$ from $\rightarrow$.
\end{definition}

At this point, we should explain the behavior of a Constraint Automata $\mathcal{A}$. Before being able to do so, we must introduce a new notion. We call $\Gamma : V \rightarrow D\cup\{*\}$ the assignment map defined on the set of all free variables of $\mathcal{A}$. Given $v \in V$, $\Gamma(v)$ is either the symbol $*$ or a value from the domain $D$. We say that $v$ is undefined if $\Gamma(v)=*$. Otherwise, $\Gamma(v) \in D$ is a value of the free variable v.
Note that the context carried by $\Gamma$ can be modified either by a local assignment (a constraint of the shape $v=d$ where $d\in D$).
%Note that there is two kinds of variables $v$ : local or externally, by the environment (explained later, after defining composition). If, $\Gamma(v)=*$ and the taken transition contains an equality $v=d$ where $d \in D $, then the map $\Gamma$ is updated such that $\Gamma(v)=d$ (the value d is assigned to v). 
%Since variables could be shared between constraints, updating a variable in a transition the environment assigns a value to the variable. either the constraint on a transition contains an equality $v=d$ where $d \in D$, or the environment on a state fixes a value for $v$ (for instance, reading acquisition from a sensor). We define as interface the set of variable modifiable by the environment.

It is now possible to express the behavior of the automaton in terms of sequence of constraints. Starting in the initial state $q_0$ with an assignment map $\Gamma$, all outgoing transition are evaluated within $\Gamma$ : for all free variable $v$ such that $\Gamma(v) \not = *$, we substitute the value $\Gamma(v)$ for $v$ in the constraint. We then get a set of outgoing transitions whose constraints are satisfiable by $\Gamma$. We choose one constraint non deterministically, and update the assignment map $\Gamma$. We repeat this procedure in the next states.




%\noindent \textbf{Constraint Automata.}

\begin{example}
%	$\mathcal{A} = \left\langle Q, \rightarrow, C, q_{0}\right\rangle$ is the automata represented in Fig. \ref{moveCA} where $Q = \{q_E,q_W\}$ and $C = \{west,east,stay_{lon}\} $. The set of variables for the constraint automaton $\mathcal{A}$ is $\mathcal{V} = \cup_{c \in C} V_c$. Each state $q \in Q$ contains a set of value assignment for each variable $v \in \mathcal{V}$ called \textit{context}. Thus, a transition is activated if its constraint is satisfiable in the context of its pre state's. Note that it is possible that the context in a state satisfies multiple constraints. In this case, the choice of the transition is non-deterministic.
The automaton represents the behavior of a system oscillating between east and west non deterministically. From state $q_W$, three actions are possible : $west, stay_{lon} $ and $east$. If all three actions are allowed (their evaluation is true), then the choice is non deterministic. If the evaluation defines a subset $c \subset C$ of true predicate, the action is chosen non deterministically over this subset $c$. Constraint automata does not let us to express an order among the constraint of a subset $c \subset C$. One possible behavior of this automaton can be described by the infinite stream of constraints $\langle east,west,stay_{lon},stay_{lon},east,west \rangle ^{\omega}$

	\begin{figure}[H]
		\centering
		\resizebox{8cm}{!}{  \begin{tikzpicture}[>=latex,shorten >=1pt,node distance=3cm,on grid,auto, node/.style={circle,draw,minimum size=25pt}, ]

  \node[state] (q0) at (-40pt,0pt) {$q_W$};
  \node[state, right = of q0] (q1) {$q_E$};
  \draw[<-,text=white] (q0) -- node[] {} ++(0,1);
  \draw[->] (q1) to[out=200,in=-20] node[below] {west} (q0);
  \draw[->] (q0) to[out=20,in=160] node[above] {east} (q1);
  \draw[->] (q1) to[out=20,in=-20,looseness=8] node[right, align=left] {east\\stay$_{lon}$} (q1);
  \draw[->] (q0) to[out=160,in=-160,looseness=8] node[left, align=left] {west\\ stay$_{lon}$} (q0);
  \end{tikzpicture}
 
}
		\caption{Constraint automata of a moving agent}
		\label{moveCA}
	\end{figure}
\end{example}


Specifying a system directly as a single Constraint Automata is not easy, since the number of states and transitions can become quite large. Instead, we want to see our system as a set of Constraint Automata in composition. We next define a composition operator for CA.

\begin{definition}
	Given $A_1=\left\langle Q_1, \rightarrow_1, C_1, q_{01}\right\rangle$ and $A_2=\left\langle Q_2, \rightarrow_2, C_2, q_{02} \right\rangle$ two constraints automata, we define the product $ A_1 \times A_2 = \left \langle Q, \rightarrow, C, (q_{01},q_{02}) \right \rangle$ where: 
	\begin{list}{-}{ }
		\item $Q= Q_1 \times Q_2 $ is a set of state and $(q_{01},q_{02})\in$ Q is the initial state.
		\item The transition relation $\rightarrow$ is the smallest relation satisfying
		$$
		\frac{q_{01} \xrightarrow{c_1}_1 q_{01}' ,\quad q_{02} \xrightarrow{c_2}_2 q_{02}'}{(q_{01},q_{02}) \xrightarrow{c_1 \land c_2}(q_{01}',q_{02}')}
		$$
		%A transition is denoted by $\left\langle q_i, c, s, p_i \right\rangle \in \rightarrow$
	\end{list}
\end{definition}

%The behavior of the composed automaton $\mathcal{A}= \mathcal{A}_1 \times \mathcal{A}_2$ can be seen as follow. 
The assignment map $\Gamma$ of $\mathcal{A}$ is the union of the assignment map $\Gamma_1$ of $\mathcal{A}_1$ and $\Gamma_2$ of $\mathcal{A}_2$. Which reflect the interaction between free variables shared between $\mathcal{A}_1$ and $\mathcal{A}_1$. 
Another illuminating view of the behavior of Constraint Automata in composition is given by thinking of concurrent executions. Both composite automaton tries to satisfy concurrently and consistently a constraint. In the case they succeed, they both take their transition, update the assignment map $\Gamma$ and perform this new concurrent execution on the new state.
It is possible to encode independent progress on each state $q\in Q$ on $\mathcal{A}$ by adding a transition $(q,\phi,q)$ where $\phi$ represents the constraint where no free variables of the automaton are defined $\bigwedge_{v \in V} (v\not=*)$.

\begin{example}
	Example of composition
%	\begin{figure}[H]
%		\centering
%		\resizebox{8cm}{!}{  \begin{tikzpicture}[>=latex,shorten >=1pt,node distance=3cm,on grid,auto, node/.style={circle,draw,minimum size=25pt}, ]

  \node[state] (q0) at (-40pt,0pt) {$q_W$};
  \node[state, right = of q0] (q1) {$q_E$};
  \draw[<-,text=white] (q0) -- node[] {} ++(0,1);
  \draw[->] (q1) to[out=200,in=-20] node[below] {west} (q0);
  \draw[->] (q0) to[out=20,in=160] node[above] {east} (q1);
  \draw[->] (q1) to[out=20,in=-20,looseness=8] node[right, align=left] {east\\stay$_{lon}$} (q1);
  \draw[->] (q0) to[out=160,in=-160,looseness=8] node[left, align=left] {west\\ stay$_{lon}$} (q0);
  \end{tikzpicture}
 
}
%		\caption{Constraint automata of a moving agent}
%		\label{moveCA}
%	\end{figure}
\end{example}











%A constraint $c$ is satisfiable if there exist an assignment $\delta : V_{c}\rightarrow D$ such that $\delta \models c$. The choice of variable name in the construction of formula will determine whether a composed constraint can be satisfied or not. 

%\begin{proposition}
	If $V_{c_1} \cap V_{c_2} = \emptyset$, $\delta_1\models c_1$ and $\delta_2 \models c_2$ then $\delta_1,\delta_2 \models c_1 \land c_2$
\end{proposition}


%In the case where the sets of free variables of two composed constraints intersect, the constraints are said to be in interactions and could no longer be satisfiable. If $c_1 := x=4 $ and $c_2 := x>5 $, the composed constraint $c := c_1 \land c_2 := x=4 \land x>5 $ is not satisfiable.
%\begin{example} Lets note $\underline{x}$ as a port variable, and build a constraint $c := \underline{a} = \underline{b}$.
%\end{example}

%
\begin{definition}
	A Constraint Automaton is a tuple $\left\langle Q, \rightarrow, C, q_{0}, \right\rangle$ where: 
	\begin{list}{-}{ }
		\item Q is a set of state and $q_0\in$ Q is the initial state.
%		\item C is a set of constraints, where a constraint is a predicate $c: (V_c \rightarrow D) \rightarrow \{\top,\bot\}$.
		\item C is a set of constraints.
		\item ${\rightarrow} \subseteq Q \times C \times Q$ is a finite relation called transition relation.
	\end{list}
		The tuple $(q,c,q') \in \rightarrow$ represents a transition from state q to state q' subject to the constraint c. We write $q \xrightarrow{c} q'$ instead of $(q,c,q') \in \rightarrow$. If c $\equiv \bot$, we can drop $q \xrightarrow{c} q'$ from $\rightarrow$.
\end{definition}

At this point, we should explain the behavior of a Constraint Automata $\mathcal{A}$. Before being able to do so, we must introduce a new notion. We call $\Gamma : V \rightarrow D\cup\{*\}$ the assignment map defined on the set of all free variables of $\mathcal{A}$. Given $v \in V$, $\Gamma(v)$ is either the symbol $*$ or a value from the domain $D$. We say that $v$ is undefined if $\Gamma(v)=*$. Otherwise, $\Gamma(v) \in D$ is a value of the free variable v.
Note that the context carried by $\Gamma$ can be modified either by a local assignment (a constraint of the shape $v=d$ where $d\in D$).
%Note that there is two kinds of variables $v$ : local or externally, by the environment (explained later, after defining composition). If, $\Gamma(v)=*$ and the taken transition contains an equality $v=d$ where $d \in D $, then the map $\Gamma$ is updated such that $\Gamma(v)=d$ (the value d is assigned to v). 
%Since variables could be shared between constraints, updating a variable in a transition the environment assigns a value to the variable. either the constraint on a transition contains an equality $v=d$ where $d \in D$, or the environment on a state fixes a value for $v$ (for instance, reading acquisition from a sensor). We define as interface the set of variable modifiable by the environment.

It is now possible to express the behavior of the automaton in terms of sequence of constraints. Starting in the initial state $q_0$ with an assignment map $\Gamma$, all outgoing transition are evaluated within $\Gamma$ : for all free variable $v$ such that $\Gamma(v) \not = *$, we substitute the value $\Gamma(v)$ for $v$ in the constraint. We then get a set of outgoing transitions whose constraints are satisfiable by $\Gamma$. We choose one constraint non deterministically, and update the assignment map $\Gamma$. We repeat this procedure in the next states.




%\begin{note}	It is possible to encode independent progress on each state $q\in Q$ by adding a transition $(q,\phi,q)$ where $\phi$ represents the constraint where no ports involve on outgoing transitions fire.\end{note}

%\begin{example}
%	$\mathcal{A} = \left\langle Q, \rightarrow, C, q_{0}\right\rangle$ is the automata represented in Fig. \ref{moveCA} where $Q = \{q_E,q_W\}$ and $C = \{west,east,stay_{lon}\} $. The set of variables for the constraint automaton $\mathcal{A}$ is $\mathcal{V} = \cup_{c \in C} V_c$. Each state $q \in Q$ contains a set of value assignment for each variable $v \in \mathcal{V}$ called \textit{context}. Thus, a transition is activated if its constraint is satisfiable in the context of its pre state's. Note that it is possible that the context in a state satisfies multiple constraints. In this case, the choice of the transition is non-deterministic.
The automaton represents the behavior of a system oscillating between east and west non deterministically. From state $q_W$, three actions are possible : $west, stay_{lon} $ and $east$. If all three actions are allowed (their evaluation is true), then the choice is non deterministic. If the evaluation defines a subset $c \subset C$ of true predicate, the action is chosen non deterministically over this subset $c$. Constraint automata does not let us to express an order among the constraint of a subset $c \subset C$. One possible behavior of this automaton can be described by the infinite stream of constraints $\langle east,west,stay_{lon},stay_{lon},east,west \rangle ^{\omega}$

	\begin{figure}[H]
		\centering
		\resizebox{8cm}{!}{  \begin{tikzpicture}[>=latex,shorten >=1pt,node distance=3cm,on grid,auto, node/.style={circle,draw,minimum size=25pt}, ]

  \node[state] (q0) at (-40pt,0pt) {$q_W$};
  \node[state, right = of q0] (q1) {$q_E$};
  \draw[<-,text=white] (q0) -- node[] {} ++(0,1);
  \draw[->] (q1) to[out=200,in=-20] node[below] {west} (q0);
  \draw[->] (q0) to[out=20,in=160] node[above] {east} (q1);
  \draw[->] (q1) to[out=20,in=-20,looseness=8] node[right, align=left] {east\\stay$_{lon}$} (q1);
  \draw[->] (q0) to[out=160,in=-160,looseness=8] node[left, align=left] {west\\ stay$_{lon}$} (q0);
  \end{tikzpicture}
 
}
		\caption{Constraint automata of a moving agent}
		\label{moveCA}
	\end{figure}
\end{example}



%Composition of constraint automata as defined in \cite{BSAR06} leads to a new constraint automaton. 

%Specifying a system directly as a single Constraint Automata is not easy, since the number of states and transitions can become quite large. Instead, we want to see our system as a set of Constraint Automata in composition. We next define a composition operator for CA.

\begin{definition}
	Given $A_1=\left\langle Q_1, \rightarrow_1, C_1, q_{01}\right\rangle$ and $A_2=\left\langle Q_2, \rightarrow_2, C_2, q_{02} \right\rangle$ two constraints automata, we define the product $ A_1 \times A_2 = \left \langle Q, \rightarrow, C, (q_{01},q_{02}) \right \rangle$ where: 
	\begin{list}{-}{ }
		\item $Q= Q_1 \times Q_2 $ is a set of state and $(q_{01},q_{02})\in$ Q is the initial state.
		\item The transition relation $\rightarrow$ is the smallest relation satisfying
		$$
		\frac{q_{01} \xrightarrow{c_1}_1 q_{01}' ,\quad q_{02} \xrightarrow{c_2}_2 q_{02}'}{(q_{01},q_{02}) \xrightarrow{c_1 \land c_2}(q_{01}',q_{02}')}
		$$
		%A transition is denoted by $\left\langle q_i, c, s, p_i \right\rangle \in \rightarrow$
	\end{list}
\end{definition}

%The behavior of the composed automaton $\mathcal{A}= \mathcal{A}_1 \times \mathcal{A}_2$ can be seen as follow. 
The assignment map $\Gamma$ of $\mathcal{A}$ is the union of the assignment map $\Gamma_1$ of $\mathcal{A}_1$ and $\Gamma_2$ of $\mathcal{A}_2$. Which reflect the interaction between free variables shared between $\mathcal{A}_1$ and $\mathcal{A}_1$. 
Another illuminating view of the behavior of Constraint Automata in composition is given by thinking of concurrent executions. Both composite automaton tries to satisfy concurrently and consistently a constraint. In the case they succeed, they both take their transition, update the assignment map $\Gamma$ and perform this new concurrent execution on the new state.
It is possible to encode independent progress on each state $q\in Q$ on $\mathcal{A}$ by adding a transition $(q,\phi,q)$ where $\phi$ represents the constraint where no free variables of the automaton are defined $\bigwedge_{v \in V} (v\not=*)$.

%The composition of two constraint automata blindly compose constraints together. From the composed set of constraints, we can identified some formula that can never be satisfied (i.e. identify to $\bot$). In this case, the constraint and its associated transition can be removed from the automaton.


\subsection{Preferences}

%The need for preferences can be justify by some simple example. If we look at the example of Fig. 1, on the state $q_W$, three different transitions are allowed : from $q_W$ to $q_E$ with constraint $east$, from $q_W$ to $q_W$ with constraint $west$ and from $q_W$ to $q_W$ with constraint $stay_{lon}$. If all of those three constraints are satisfiable, we will pick one non deterministically. But what if we want, in this case, to determine which one we prefer ? How could we encode that we prefer taking a transition instead of another one ?
Semirings constitutes a suitable mathematical structure to define the notion of preferences. A constraint semiring \cite{B04} induces an order relation among its element. Therefore, following the initial example on stream and behavior, each action of the stream is attached a preference in a c-semiring. The csemiring will serve as algebraic structure to order streams. We will look at the composition of set of streams with preferences, and the resulting stream order. 
Given two systems $A_1$ and $A_2$ with ordered stream and synchronization constraints ($A_1$ must synchronize along its run with $A_2$), what is the ordered stream generated by the parallel composition of $A_1$ and $A_2$ ? To answer this question, we must study the properties of the csemiring and order induced by the composition of different csemirings. 

A csemiring has an induced order defined given by the properties of its two operators : $+$ and $\times$. The $+$ operator can intuitively be seen as choosing the best value between two value. Thus, if $a,b \in E$, $a+b=a$ can be interpreted as $a$ "is preferred" to $b$. Alternatively, $\times$ is intuitively used to compose preferences. Then $a \times b$ is a new preference, which could be different from $a$ or $b$.

%As we want to compose ordered stream, we 
Since csemiring define an order among its element, and we have "intuition" behind its operators, we now want to look at the composition of different csemiring, and get intuition behind the order defined on the composed csemiring. Recall that we also want the composition to be ordered, thus being described by a csemiring.

On the search of a structure for our product of csemiring that :
\begin{list}{-}{}
	\item composed preferences of composite actions.
	\item induces an order on the composition.
	\item can later on be composed with a new csemiring.
\end{list}

The carrier of the semiring should contain any possible elements from the composite csemiring. Then, we call the free product such structure, being the cartesian product of all csemiring involved in the composition. It is possible to show that the direct product of two csemiring is itself a csemiring, where $+$ and $\times$ operator are interpreted as $+$ and $\times$ of the underlying csemiring. Thus, if $A_1$ compose with $A_2$, where preferences of $A_1$ are defined over a csemiring $E_1 \times E_2$ and preferences of $A_2$ are defined over $E_3$, we get $A = A_1 \times A_2$ with preferences defined over $E_1 \times E_2 \times E_3$.

At this point, remark that the definition of the composition is not without implication. Recall the csemiring induces an order among its element. By defining the product of csemiring as the cartesian product with interpretation of $\times$ and $+$ as underlying csemiring operators, we also define a certain order among the generated streams of the composed system. In this case, the induced order is the following : chose the best preference of the left csemiring, and chose the best preference of the right csemiring, and the resulting preference is the tuple consisting of those two preferences. 

\begin{example}
	The case where $\langle west, east \rangle ^{\omega}$ is a possible behavior of the system, and are attached some preferences : $\langle (west ,2), (east,5) \rangle ^{\omega}$ where $2$ and $5$ are value of the weighted csemiring. The stream described by $\langle (stay_{lon} ,3) \rangle ^{\omega}$ is also a possible behavior, but since $2<3$, the action $west$ is preferred to the action $stay_{lon}$, therefore the behavior described by the stream $\langle (west ,2), (east,5) \rangle ^{\omega}$ is preferred to the behavior described by the stream $\langle (stay_{lon} ,3) \rangle ^{\omega}$. However, taking one step further, in the next element of the stream, we now compare $\langle (east,5),(west ,2) \rangle ^{\omega}$ with $\langle (stay_{lon} ,3) \rangle ^{\omega}$. Since $3<5$, the stream order changed, and the first behavior (which was $\langle (east,5),(west ,2) \rangle ^{\omega}$) is no more preferred to the other behavior  $\langle (stay_{lon} ,3) \rangle ^{\omega}$. The system will now prefer to take the action $stay_{lon}$. Repeating this process, we finally get the stream $\langle (west,2),(stay_{lon},3) \rangle ^{\omega}$ as a preferred behavior.
\end{example}

\begin{example}
	Suppose a composition with a new behavior $\langle (snap,0.5) \rangle ^{\omega}$. By definition, the composed behavior are $\langle ((west,snap),(2,0.5)),((east,snap),(5,0.5)) \rangle ^{\omega}$ and $\langle ((stay_{lon},\emptyset),(3,\emptyset)) \rangle ^{\omega}$.
\end{example}




\begin{definition} (semiring)
	A semiring is a non empty set $E$ on which operations of addition and multiplication have been defined such that:
	\begin{list}{-}{ }
		\item $(E,+)$ is a commutative monoid with identity element 0.
		\item $(E,\times)$ is a monoid with identity element 1.
		\item Multiplication distributes over addition from either sides.
		\item $0 \times e = e \times 0 = 0$ for all $ e \in E$
	\end{list} 
	We note $\left\langle E, +, \times, \boldmath{1},\boldmath{0} \right\rangle$ to refer to this semiring.
\end{definition}

\begin{definition} (csemiring)
	A csemiring is a semiring  $\left\langle E, +, \times, \boldmath{1},\boldmath{0} \right\rangle$ with additional properties :
	\begin{list}{-}{ }
		\item + is idempotent. We use the notation $\sum(A)$ in prefix notation to describe the sum of all elements of a possibly infinite set $A \subset E$.
		\item $\times$ is commutative.
		\item $1 + e = e + 1 = 1$ for all $ e \in E$
	\end{list} 
	A csemiring admits a partial order $\leq_{E}$, defined as the smallest relation satisfying :
	$$
	\frac{e,e' \in E \quad e+e' = e}{e \leq_{E} e'}
	$$
\end{definition}

It is shown in [2] that $\leq$ satisfies the following properties:
\begin{list}{-}{ }
	\item $\leq$ is a partial order, with minimum 0 and maximum 1;
	\item $x + y$ is the least upper bound of x and y;
	\item $x \times y$ is a lower bound of x and y;
	\item (S, $\leq $) is a complete lattice (i.e., the greatest lower bound exists);
	\item + and $\times $ are monotone on $\leq$.
	\item if $\times $ is idempotent, then + distributes over $\times $, $x \times y$ is the greatest lower bound of $x$ and $y$, and (S, $\leq $) is a distributive lattice.
\end{list} 
%\begin{definition} (c-semiring)
%	A constraint semiring is a tuple $\left\langle \mathbb{E}, \bigvee, \otimes, \boldmath{1},\boldmath{0} \right\rangle$ where, for all $e\in E$, $E' \subset \mathbb{E}$, $\mathcal{E} \in 2^{\mathbb{E}}$, the following holds  :
%	\begin{list}{-}{ }
%		\item $\mathbb{E}$ is a set with $\boldmath{1},\boldmath{0} \in \mathbb{E}$
%		\item $\otimes : \mathbb{E} \times \mathbb{E} \rightarrow \mathbb{E}$ is a commutative and associative operator, with $\boldmath{0}_{\mathbb{E}} \otimes e = \boldmath{0}_{\mathbb{E}}$ and $\boldmath{1}_{\mathbb{E}} \otimes e = e$
%		\item $\bigvee : 2^{\mathbb{E}} \rightarrow \mathbb{E}$ , with $\bigvee{\mathbb{E}}=\boldmath{1} $, $\bigvee{\emptyset}=\boldmath{0}$ and $\bigvee_{E \in \mathcal{E}}{(\bigvee{E})}= \bigvee{(\bigcup \{E | E \in \mathcal{E}\})}$
%		\item $\otimes$ distributes over $\bigvee$ with $e \otimes \bigvee E = \bigvee \{ e \otimes e' | e' \in E \}$
%	\end{list} 
%	A c-semiring induces an order $\leq_{\mathbb{E}}$, defined as the smallest relation satisfying :
%	$$
%	\frac{e,e' \in \mathbb{E} \quad \bigvee\{e,e'\}=e}{e \leq_{\mathbb{E}} e'}
%	$$
%\end{definition}

Intuitively, the $\times$ operator behaves as a composition operator for preferences. The $\sum$ can be seen as the choice of the best preference over a set of preferences, where $1$ is the highest preference and $0$ is the lowest. We give some well known instances of c-semirings.

\begin{example}
Examples of c-semirings :
\begin{list}{-}{}
	\item The Boolean semiring $\mathbb{B}$:	$\left\langle \{\top,\bot\}, \lor, \land, \boldmath{1}_{\mathbb{B}}, \boldmath{0}_{\mathbb{B}} \right\rangle$ where $\boldmath{0}_{\mathbb{B}}=\bot$ and $\boldmath{1}_{\mathbb{B}}=\top$
	
	\item The Weighted semiring $\mathbb{W}$ :$\left\langle \mathbb{N}\cup\{\infty\}, \min, +, \boldmath{1}_{\mathbb{W}}, \boldmath{0}_{\mathbb{W}} \right\rangle$ where $\boldmath{0}_{\mathbb{W}}=\infty$ and $\boldmath{1}_{\mathbb{B}}=0$	
\end{list}

\paragraph{Application}
In the case of our trekker, he first uses boolean semiring to model his choice. If we $left$ and $right$ are propositional variable, we could model the trekker's choice by the following value :
$$ 	TrekChoice = left \lor right $$
Without any information, he can either chose left (make left true) or right.

Now that he has access to the distance, he can chose a weighted semiring instead, and model his choice by the following function :
$$
	TrekChoice = 	\left\{
	\begin{array}{rl}
	left \quad & \text{ if } l+_Wr= min(l,r) = l \\
	right \quad & \text{ otherwise }
	\end{array}
	\right.
$$
where $l$ is the time it takes on the left path, and $r$ on the right path.
%\paragraph{Example}
%\begin{list}{-}{}
%	\item The Boolean semiring $\mathbb{B}$:
%	$$\left\langle \{\top,\bot\}, \lor, \land, \boldmath{1}_{\mathbb{B}}, \boldmath{0}_{\mathbb{B}} \right\rangle$$ where $\boldmath{0}_{\mathbb{B}}=\bot$ and $\boldmath{1}_{\mathbb{B}}=\top$
%	Remarque : boolean semiring is a structure for valuation of first order logic sentences.
	
%	\item The Weighted semiring $\mathbb{W}$ :
%	$$\left\langle \mathbb{N}\cup\{\infty\}, \min, \max, \boldmath{1}_{\mathbb{W}}, \boldmath{0}_{\mathbb{W}} \right\rangle$$ where $\boldmath{0}_{\mathbb{W}}=\infty$ and $\boldmath{1}_{\mathbb{B}}=0$	
%\end{list}
\end{example}



%\input{subsections/Preferences/}


%For this section, we assume $E=\langle E, \times,+,1,0 \rangle$ and $F=\langle F, \times,+,1,0 \rangle$ two c-semirings. 

\begin{definition} We define the disjoint union of two sets A and B while identifying elements of the intersection S:	
	$$A \cup_{S}B = (A \uplus B)/\sim, \quad x \sim y \Leftrightarrow x=y,\quad \text{where } y\in S$$
\end{definition}


Free(X) is the smallest set such that :
\begin{enumerate}
	\item $X \subseteq Free(X)$
	\item $x,y \in Free(X) \Rightarrow x \otimes y \in Free(X)$
	\item $x,y \in Free(X) \Rightarrow x + y \in Free(X)$
\end{enumerate}

We define ${\equiv} \subset Free(X)^2$ as the smallest congruence with regard to $\otimes,+$:
\begin{enumerate}
	\item $x \otimes y \equiv y \otimes x$
	\item $(x+x')\otimes y \equiv x \otimes y + x' \otimes y$
	\item $x+x \equiv x$
	\item $(x \otimes y) \otimes z \equiv x \otimes (y \otimes z)$
\end{enumerate}


\begin{definition} Given E,F csemiring and $\mathbb{B}$ the boolean csemiring, the tensor product  $E\otimes_{\mathbb{B}}F$ defined by the tuple $\langle Free(E \cup_{\mathbb{B}} F) /\equiv, +, \otimes, 0_{\mathbb{B}}, 1_{\mathbb{B}} \rangle $ is the co-product of E and F, where $0_{\mathbb{B}} \otimes a = 0_{\mathbb{B}}$, $1_{\mathbb{B}} + a = 1_{\mathbb{B}}$, $0_{\mathbb{B}} + a = a$ and $1_{\mathbb{B}} \otimes a = a$ for all $a \in E\otimes_{\mathbb{B}}F$
\end{definition}


%Since + is idempotent in a csemiring, the tensor product needs only be defined over a Boolean semiring. There exists a natural embedding of csemiring E and F into the tensor product $E \otimes_{\mathbb{B}} F$ with the maps $i_E : E \rightarrow E \otimes_{\mathbb{B}} F, e \longmapsto e $ and $i_F : F \rightarrow E \otimes_{\mathbb{B}} F, f \longmapsto f  $.
%\begin{figure}[H]
%	\centering
%	\resizebox{8cm}{!}{ \begin{tikzpicture}[>=latex,shorten >=1pt,node distance=3cm,on grid,auto, node/.style={circle,draw,minimum size=25pt}, ]
 
 \node[draw=none] (q0) at (0pt,0pt) {$G$};
 \node[draw=none, right = of q0] (q1) {$F$};
 \node[draw=none, left = of q0] (q2) {$E$};
 \node[draw=none, above = of q0] (q3) {$E \otimes_{\mathbb{B}} F$};
 \draw[->] (q1) to node[above right] {$i_F$} (q3);
 \draw[->] (q2) to node[above left] {$i_E$} (q3);
 \draw[->] (q1) to node[above] {f} (q0);
 \draw[->] (q2) to node[above] {e} (q0);
 \draw[dashed,->] (q3) to node[right] {g} (q0);
 \end{tikzpicture}
}
%	\caption{Coproduct universal property}
%	\label{coprod}
%\end{figure}

%We are interested in quotients of $E \otimes_{\mathbb{B}} F$ as definition of different products of csemiring.

\begin{definition}(lexicographic) A quotient $h: E \otimes_{\mathbb{B}} F \rightarrow L$ is lexicographic if and only if, for all $e_1,e_2 \in E$ and $f_1,f_2 \in F$, there exists $i \in \{1,0\}$, such that we have:
	$$h(e_1 \times f_1 + e_2 \times f_2)= h(e_i \times f_i)$$
	whenever $e_1+e_2=e_i$ and $e_1 \not = e_2$; or $f_1 + f_2 = f_i$ and $e_1=e_2$
\end{definition}
\noindent
\begin{definition}
	(collapsing elements) We define the set of collapsing element of a semiring E by $$\mathcal{C}(E) = \{e \in E \quad | \quad \exists e_1 e_2 \in E, \quad e_1 \times e = e_2 \times e \land e_1 \not = e_2  \}$$
\end{definition}
In other words, it is possible to break a strict inequality after multiplication with a collapsing element. Collapsing element does not preserve strict inequality. Therefore, in the case of some product semiring (for instance lexicographic), distribution of $\times$ over + does not hold. When we come to define product, we should be aware of collapsing element in order to get back a csemiring. \\

Example : E, F csemirings. $e,e_1,e_2 \in E$ such that $e_1 \times e = e_2 \times e \land e_1 < e_2$ and $f_1,f_2 \in F$ with $f_2<f_1 $. We look at a possible element of the coproduct : $e_1 \times f_1 + e_2 \times f_2$. Assuming we have a quotient, and $h : E \sqcup F \rightarrow E \times_l F$ can project this term in a lexicographic product space, we would get $h(e_1 \times f_1 + e_2 \times f_2) = (e_2,f_2)$. By distributivity law, and given the homomorphic properties of h, we have :
\begin{align*}
	&h(e \times (e_1 \times f_1 + e_2 \times f_2)) = h(e \times e_1 \times f_1 + e \times e_2 \times f_2) = h(e \times e_2 \times f_1) \quad \\ \text{and} \\
	 &h(e) \times h(e_1 \times f_1 + e_2 \times f_2) = h(e) \times h( e_2 \times f_2) = h(e \times e_2 \times f_2) \\ \text{Thus} \\
	 & h(e \times (e_1 \times f_1 + e_2 \times f_2)) \not= h(e) \times h(e_1 \times f_1 + e_2 \times f_2)
\end{align*}

\begin{lemma}
	If E is cancellative, then there exists a lexicographic quotient $h : E \otimes_{\mathbb{B}} F \rightarrow L$
	%a relation $\equiv$ on $E \otimes_{\mathbb{B}} F$ such that for all $e_1,e_2 \in E$, $f_1,f_2 \in F$, there exists $i \in \{1,0\}$, such that we have :$$e_1 \times f_1 + e_2 \times f_2 \equiv e_i \times f_i$$ 	whenever $e_1+e_2=e_i$ and $e_1 \not = e_2$; or $f_1 + f_2 = f_i$ and $e_1=e_2$
\end{lemma}
%\begin{lemma}
%	The boolean csemiring is lexicographic
%\end{lemma}
%%\begin{definition}
%	C-semiring are R-algebra where R is the unital ring.
%\end{definition}

%The co-product of two csemiring E and F consists in the disjoint union of the elements of E, F and the tensor product $E \otimes_{\mathbb{B}} F$ over the Boolean semiring.
%\begin{definition}
%	The co-product $E \sqcup F$ of two c-semirings $E$ and $F$ is defined by 
%	$E \oplus F \oplus E \otimes_{\mathbb{B}} F$
%	modulo the equations $x + 1 = 1$ and $x \otimes 0 = 0$, for all x $\in E\sqcup F$.
%\end{definition}
% We note $\lfloor \cdot \rfloor : E \rightarrow E \sqcup F$ for the right injection and $\lceil \cdot \rceil : F \rightarrow E \sqcup F$ for the left injection.
\begin{theorem}
	Let E and F be csemiring. Then, given any csemiring G and maps $e : E \rightarrow G $ and $f : F \rightarrow G $, there exists a unique map $g : E \sqcup F \rightarrow G $ such that $g \circ \lfloor \cdot \rfloor  = e$ and $g \circ \lceil \cdot \rceil = f$
\end{theorem}

\begin{proof}
	By universal property of the co-product, there exists a unique map $g:E\sqcup F \rightarrow G$ such that the following diagram commutes:
	\begin{figure}[H]
		\centering
		\resizebox{8cm}{!}{ \begin{tikzpicture}[>=latex,shorten >=1pt,node distance=3cm,on grid,auto, node/.style={circle,draw,minimum size=25pt}, ]
 
 \node[draw=none] (q0) at (0pt,0pt) {$G$};
 \node[draw=none, right = of q0] (q1) {$F$};
 \node[draw=none, left = of q0] (q2) {$E$};
 \node[draw=none, above = of q0] (q3) {$E \otimes_{\mathbb{B}} F$};
 \draw[->] (q1) to node[above right] {$i_F$} (q3);
 \draw[->] (q2) to node[above left] {$i_E$} (q3);
 \draw[->] (q1) to node[above] {f} (q0);
 \draw[->] (q2) to node[above] {e} (q0);
 \draw[dashed,->] (q3) to node[right] {g} (q0);
 \end{tikzpicture}
}
		\caption{Coproduct universal property}
		\label{coprod}
	\end{figure}
\end{proof}

\begin{lemma}
	Given E a csemiring, there exists a map $g : E \sqcup E \rightarrow E$.
\end{lemma}
\begin{proof}
	property of the co-product.
\end{proof}

%\noindent
%\textbf{Properties of composition} 

\noindent
\textbf{collapsing element and coproduct} We define the set of collapsing element of a semiring E by $$\mathcal{C}(E) = \{e \in E \quad | \quad \exists e_1 e_2 \in E, \quad e_1 \times e = e_2 \times e \land e_1 < e_2  \}$$
In other words, it is possible to break a strict inequality after multiplication with a collapsing element. Collapsing element does not preserve strict inequality. Therefore, in the case of some product semiring (for instance lexicographic), distribution of $\times$ over + does not hold. When we come to define product, we should be aware of collapsing element in order to get back a csemiring. \\

Example : E, F csemirings. $e,e_1,e_2 \in E$ such that $e_1 \times e = e_2 \times e \land e_1 < e_2$ and $f_1,f_2 \in F$ with $f_2<f_1 $. We look at a possible element of the coproduct : $e_1 \otimes f_1 \oplus e_2 \otimes f_2$. Assuming we have a quotient, and $h : E \sqcup F \rightarrow E \times_l F$ can project this term in a lexicographic product space, we would get $h(e_1 \otimes f_1 \oplus e_2 \otimes f_2) = (e_2,f_2)$. However, now we would have the following :
$$
h(e \otimes e_1 \otimes f_1 \oplus e \otimes e_2 \otimes f_2) = (e \otimes_E e_2,f_1)  \not= (e \otimes_E e_2,f_2) = (e,1)\otimes_l (e_2,f_2) = h(e) \otimes_l h(e_1\otimes f1 \oplus e_2 \otimes f_2)
$$

%For the following definition, we restrict the carrier of a csemiring to its non-cancellative elements if we want the product to be a csemiring.

%We want to be able to define the following. Let E, F and C be three csemirings and $f: E \rightarrow C$, $g : F \rightarrow C $

%\begin{definition}
%	E and F csemirings, the co-product $E \sqcup F$ is defined by $E \oplus F \oplus E \otimes_{\mathbb{N}} F$ modulo $x \oplus 1 = 1 $ and $x \otimes 0 = 0 $.
%	\begin{list}{-}{ }
%		\item $x \oplus 1 = 1 $
%		\item $x \otimes 0 = 0 $
%	\end{list}
%\end{definition}

%\begin{definition}
%Given E and F csemirings, let $i_e : E \rightarrow E \otimes_{B} F$ and $i_f : F \rightarrow E \otimes_{B} F$ we define the co-product $E \otimes_{B} F$ where $B=\{0,1\}$ (since + is idempotent).
%\end{definition}


%\begin{definition}
%The co-product $\mathbb{E} \sqcup \mathbb{F}$ of two c-semirings $\mathbb{E}$ and $\mathbb{F}$ is defined by 
%$\mathbb{B} \oplus \mathbb{E} \oplus \mathbb{F} \oplus \mathbb{E} \otimes_{\mathbb{B}} \mathbb{F}$
%$\mathbb{E} \otimes_{\mathbb{B}} \mathbb{F}$
%, where $ \mathbb{B}$ is the boolean c-semiring.
%modulo the equations $x + 1 = 1$ and $x \otimes 0 = 0$, for all x $\in \mathbb{E}\sqcup \mathbb{F}$.
%\end{definition}

\begin{definition}Let $\mathbb{E}$ and $\mathbb{F}$ be two c-semirings. Their composition written $\mathbb{E} \sqcup \mathbb{F} = \left\langle \mathbb{E} \sqcup \mathbb{F}, +, \times, \boldmath{1},\boldmath{0} \right\rangle$ is defined as :
	\begin{list}{-}{}
		\item The carrier is given by the co-product $\mathbb{E} \sqcup \mathbb{F}$, and we write $\lfloor \cdot \rfloor : \mathbb{E} \rightarrow \mathbb{E} \sqcup \mathbb{F}$ and $\lceil \cdot \rceil : \mathbb{F} \rightarrow \mathbb{E} \sqcup \mathbb{F}$ the left and right injections.
		\item The choice operator $+ : 2^{\mathbb{E} \sqcup \mathbb{F}} \rightarrow \mathbb{E} \sqcup \mathbb{F}$, where, given $e_1,e_2 \in \mathbb{E}$ and $s_1,s_2 \in \mathbb{F}$: : 
		\begin{align*}
			& \lceil e_1 \rceil + \lceil e_2 \rceil = \lceil e_1 + e_2 \rceil \\
			& \lfloor s_1 \rfloor + \lfloor s_2 \rfloor = \lfloor s_1 + s_2 \rfloor  %\\
			%			& \lfloor s_1 \rfloor \times \lceil e_1 \rceil + \lfloor s_2 \rfloor \times \lceil e_2 \rceil = \lfloor s_1 + s_2 \rfloor \times \lceil e_1 +e_2 \rceil
		\end{align*}
		\item The composition operator $\times : \mathbb{E} \sqcup \mathbb{F} \times \mathbb{E} \sqcup \mathbb{F} \rightarrow \mathbb{E} \sqcup \mathbb{F}$, where, given $e_1,e_2 \in \mathbb{E}$ and $s_1,s_2 \in \mathbb{F}$:
		\begin{align*}
			& \lceil e_1 \rceil \times \lceil e_2 \rceil = \lceil e_1 \times e_2 \rceil \\
			& \lfloor s_1 \rfloor \times \lfloor s_2 \rfloor = \lfloor s_1 \times s_2 \rfloor 
		\end{align*}
		\item $\textbf{0}_{\mathbb{E} \sqcup \mathbb{F}}=\lceil\textbf{0}_{\mathbb{E}}\rceil=\lfloor\textbf{0}_{\mathbb{F}}\rfloor$ and $\textbf{1}_{\mathbb{E} \sqcup \mathbb{F}}=\lceil\textbf{1}_{\mathbb{E}}\rceil=\lfloor\textbf{1}_{\mathbb{F}}\rfloor$
	\end{list}
\end{definition}


%\begin{theorem}
	Let $\mathbb{E}$ and $\mathbb{G}$ two c-semirings such that $e : \mathbb{E} \rightarrow \mathbb{G} $ is an homomorphism between $ \mathbb{E} $ and $ \mathbb{G} $. There exists a unique morphism $g :  \mathbb{E} \sqcup \mathbb{E} \rightarrow \mathbb{G}$ such that
\end{theorem}

\begin{proof} 
	We define $i_1 = \mathbb{E} \rightarrow \mathbb{E} \sqcup \mathbb{E} $ the right injection and $i_2 = \mathbb{E} \rightarrow \mathbb{E} \sqcup \mathbb{E} $ as the left injection.
	%	For this proof, we show that $\mathbb{E} \sqcup \mathbb{E} = \left\langle \mathbb{E} \sqcup \mathbb{E}, +_{\mathbb{E} \sqcup \mathbb{E}}, \times_{\mathbb{E} \sqcup \mathbb{E}}, \boldmath{1}_{\mathbb{E} \sqcup \mathbb{E}},\boldmath{0}_{\mathbb{E} \sqcup \mathbb{E}} \right\rangle =  \left\langle \mathbb{E}, +_{\mathbb{E}}, \times_{\mathbb{E}}, \boldmath{1}_{\mathbb{E}},\boldmath{0}_{\mathbb{E}} \right\rangle$. We note $\lceil \cdot \rceil : \mathbb{E} \rightarrow \mathbb{E}\sqcup\mathbb{E}$ and by definition we have $ \boldmath{1}_{\mathbb{E}\sqcup\mathbb{E}} =  \boldmath{1}_{\mathbb{E}} $ and  $ \boldmath{0}_{\mathbb{E}\sqcup\mathbb{E}} =  \boldmath{0}_{\mathbb{E}} $. The carrier is written $\mathbb{B} \oplus \mathbb{E} \oplus \mathbb{E} \oplus \mathbb{E} \otimes_{\mathbb{B}}\mathbb{E}$
\end{proof}



%We are interested in quotients of $E \otimes_{\mathbb{B}} F$ as definition of different products of csemiring.

\begin{definition}(lexicographic) A quotient $h: E \otimes_{\mathbb{B}} F \rightarrow L$ is lexicographic if and only if, for all $e_1,e_2 \in E$ and $f_1,f_2 \in F$, there exists $i \in \{1,0\}$, such that we have:
	$$h(e_1 \times f_1 + e_2 \times f_2)= h(e_i \times f_i)$$
	whenever $e_1+e_2=e_i$ and $e_1 \not = e_2$; or $f_1 + f_2 = f_i$ and $e_1=e_2$
\end{definition}
\noindent
\begin{definition}
	(collapsing elements) We define the set of collapsing element of a semiring E by $$\mathcal{C}(E) = \{e \in E \quad | \quad \exists e_1 e_2 \in E, \quad e_1 \times e = e_2 \times e \land e_1 \not = e_2  \}$$
\end{definition}
In other words, it is possible to break a strict inequality after multiplication with a collapsing element. Collapsing element does not preserve strict inequality. Therefore, in the case of some product semiring (for instance lexicographic), distribution of $\times$ over + does not hold. When we come to define product, we should be aware of collapsing element in order to get back a csemiring. \\

Example : E, F csemirings. $e,e_1,e_2 \in E$ such that $e_1 \times e = e_2 \times e \land e_1 < e_2$ and $f_1,f_2 \in F$ with $f_2<f_1 $. We look at a possible element of the coproduct : $e_1 \times f_1 + e_2 \times f_2$. Assuming we have a quotient, and $h : E \sqcup F \rightarrow E \times_l F$ can project this term in a lexicographic product space, we would get $h(e_1 \times f_1 + e_2 \times f_2) = (e_2,f_2)$. By distributivity law, and given the homomorphic properties of h, we have :
\begin{align*}
	&h(e \times (e_1 \times f_1 + e_2 \times f_2)) = h(e \times e_1 \times f_1 + e \times e_2 \times f_2) = h(e \times e_2 \times f_1) \quad \\ \text{and} \\
	 &h(e) \times h(e_1 \times f_1 + e_2 \times f_2) = h(e) \times h( e_2 \times f_2) = h(e \times e_2 \times f_2) \\ \text{Thus} \\
	 & h(e \times (e_1 \times f_1 + e_2 \times f_2)) \not= h(e) \times h(e_1 \times f_1 + e_2 \times f_2)
\end{align*}

\begin{lemma}
	If E is cancellative, then there exists a lexicographic quotient $h : E \otimes_{\mathbb{B}} F \rightarrow L$
	%a relation $\equiv$ on $E \otimes_{\mathbb{B}} F$ such that for all $e_1,e_2 \in E$, $f_1,f_2 \in F$, there exists $i \in \{1,0\}$, such that we have :$$e_1 \times f_1 + e_2 \times f_2 \equiv e_i \times f_i$$ 	whenever $e_1+e_2=e_i$ and $e_1 \not = e_2$; or $f_1 + f_2 = f_i$ and $e_1=e_2$
\end{lemma}
%\begin{lemma}
%	The boolean csemiring is lexicographic
%\end{lemma}

%\begin{proposition}
	Given $\mathbb{E}$ and $\mathbb{F}$ c-semiring, their composition defined by the co-product $\mathbb{E} \sqcup \mathbb{F}$ is a c-semiring.
\end{proposition}
\begin{proof}Properties of + and $\times$ inherited from underlying semiring.
\end{proof}

\begin{proposition} Given $\mathbb{E}$ and $\mathbb{F}$ c-semiring, for all element $s \in \mathbb{E} \sqcup \mathbb{F}$, there exist $s_1,s_3 \in \mathbb{E}$, $s_2,s_4 \in \mathbb{F}$ such that
	$$
	s = \lceil s_1 \rceil + \lfloor s_2 \rfloor + \lceil s_3 \rceil \times \lfloor s_4 \rfloor
	$$
	is a normal form.
\end{proposition}


%\subsection{Soft Constraint Automata}

\begin{definition}
	Soft Constraint Automata (SCA) is a tuple $\left\langle Q, \rightarrow, C, \mathbb{S}, q_{0}, \right\rangle$ where : 
	\begin{list}{-}{ }
		\item $Q$ is a set of state and $q_0\in$ Q is the initial state.
		\item C is a set of constraints.
		\item $\rightarrow \subseteq Q \times C \times \mathbb{S} \times Q$ is a finite relation called transition relation. 
		%A transition is denoted by $\left\langle q_i, c, s, p_i \right\rangle \in \rightarrow$
	\end{list}
\end{definition}

\begin{example}
	The automaton of Fig.2 represents the same 
	\begin{figure}[H]
		\centering
		\resizebox{8cm}{!}{ \begin{tikzpicture}[>=latex,shorten >=1pt,node distance=3cm,on grid,auto, node/.style={circle,draw,minimum size=25pt}, ]

 \node[state] (q0) at (-40pt,0pt) {$q_W$};
 \node[state, right = of q0] (q1) {$q_E$};
 \draw[<-,text=white] (q0) -- node[] {} ++(0,1);
 \draw[->] (q1) to[out=200,in=-20] node[below] {west , 5} (q0);
 \draw[->] (q0) to[out=20,in=160] node[above] {east , 5} (q1);
 \draw[->] (q1) to[out=20,in=-20,looseness=8] node[right, align=left] {east , 0\\stay$_{lon}$ , 5} (q1);
 \draw[->] (q0) to[out=160,in=-160,looseness=8] node[left, align=left] {west , 0\\ stay$_{lon}$ , 5} (q0);
 \end{tikzpicture}
}
		\caption{Soft constraint automata for moving}
		\label{moveSCA}
	\end{figure}
\end{example}


\begin{definition}
	Given $A_1=\left\langle Q_1, \rightarrow_1, C_1,\mathbb{S}, q_{01} \right\rangle$ and $A_2=\left\langle Q_2, \rightarrow_2, C_2,\mathbb{S}, q_{02} \right\rangle$ two SCA, we define the product $ A_1 \times A_2 = \left \langle Q, \rightarrow, C,\mathbb{S}, (q_{01},q_{02}) \right \rangle$ where : 
	\begin{list}{-}{ }
		\item $Q= Q_1 \times Q_2 $ is a set of state and $(q_{01},q_{02})\in$ Q is the initial state.
		\item The transition relation $\rightarrow$ is the smallest relation satisfying
		$$
		\frac{q_{01} \xrightarrow{c_1,e_1}_1 q_{01}' ,\quad q_{02} \xrightarrow{c_2,e_2}_2 q_{02}' }{(q_{01},q_{02}) \xrightarrow{c_1 \land c_2,e_1 \otimes e_2}(q_{01}',q_{02}')}
		$$
		%A transition is denoted by $\left\langle q_i, c, s, p_i \right\rangle \in \rightarrow$
	\end{list}
\end{definition}















%Compositionality is necessary to define a complex system as a composition of simpler systems. For this reason, we make sure that the following definitions respect a closure property under composition. 
%In order to get an intuition behind the definitions of this section, I suggest the following approach : as long as nothing is specified, everything is possible. We then start with an infinite system, with infinite behavior. As it will be explain, the system involves a set of variables. Without more informations, the possible behavior of the system is described by a set of sets of streams, representing the successive values of the variables assigned by the system. %We will justify each definitions by an operation on streams that restrict the possible behavior of our system to a subclass of behaviors.
%Two different visions of computation conflict at this point. Either we think of computer as a functional assignment 