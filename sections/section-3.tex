\section{Example}

\begin{example}Movement components\\
\begin{figure}[H]
	\centering
	\resizebox{8cm}{!}{\input{fig/moveSCA.tikz}}
	\caption{Soft constraint automata for moving}\label{fig:myfigure}
\end{figure}

\begin{align*}
\phi = &(west \otimes 0 \otimes q=q_W) \oplus (stay_{lon} \otimes q=q_W \otimes 5) \oplus (east \otimes 5 \otimes q=q_W \otimes q'=q_E) \oplus (east \otimes 0 \otimes q=q_E) \oplus \\
		&(stay_{lon} \otimes q=q_E \otimes 5) \oplus (west \otimes 5 \otimes q=q_E \otimes q'=q_W)
\end{align*}

\begin{figure}[H]
	\centering
	\resizebox{7cm}{!}{ \begin{tikzpicture}[>=latex,shorten >=1pt,node distance=3cm,on grid,auto, node/.style={circle,draw,minimum size=25pt}, ]

 \node[state] (q0) at (-40pt,0pt) {$q_S$};
 \node[state, right = of q0] (q1) {$q_M$};
 \node[state, right = of q1] (q2) {$q_N$};

 \draw[<-,text=white] (q1) -- node[] {} ++(0,1);
 \draw[->] (q1) to[out=200,in=-20] node[below] {south,0} (q0);
 \draw[->] (q0) to[out=20,in=160] node[above] {north, 5} (q1);
 \draw[->] (q2) to[out=200,in=-20] node[below] {south, 5} (q1);
 \draw[->] (q1) to[out=20,in=160] node[above] {north, 0} (q2);
 \draw[->] (q0) to[out=-70,in=-110,looseness=8] node[below, align=left] {stay$_{lon}$} (q0);
 \draw[->] (q1) to[out=-70,in=-110,looseness=8] node[below, align=left] {stay$_{lon}$} (q1);
 \draw[->] (q2) to[out=-70,in=-110,looseness=8] node[below, align=left] {stay$_{lon}$} (q2);
 \end{tikzpicture}
}
	\caption{Soft constraint automata for diverging}\label{fig:myfigure}
\end{figure}
\end{example}

\begin{align*}
\phi = 
\end{align*}

\begin{example}Energy management component\\
\begin{figure}[H]
	\centering
	\resizebox{12cm}{!}{ \begin{tikzpicture}[>=latex,shorten >=1pt,node distance=3cm,on grid,auto, node/.style={circle,draw,minimum size=25pt}, ]

 \node[state] (q0) at (-40pt,0pt) {$q_0$};
 \node[state, right = of q0] (q1) {$q_1$};
 \node[state, right = of q1] (q2) {$q_2$};
 \node[state, right = of q2] (q3) {$q_3$};
 \node[state, right = of q3] (q4) {$q_4$};
 \draw[<-] (q4) -- node[above right] {} ++(1,0);
 \draw[->] (q1) to[out=200,in=-20] node[below] {discharge$_1$, 2} (q0);
 \draw[->] (q0) to[out=20,in=160] node[above] {charge, 0} (q1);
 \draw[->] (q2) to[out=-110,in=-70,looseness=0.7] node[below] {discharge$_2$, 5} (q0);
 \draw[->] (q2) to[out=200,in=-20] node[below] {discharge$_1$, 2} (q1);
 \draw[->] (q1) to[out=20,in=160] node[above] {charge, 0} (q2);
 \draw[->] (q3) to[out=-110,in=-70,looseness=0.7] node[below] {discharge$_2$, 5} (q1);
 \draw[->] (q3) to[out=200,in=-20] node[below] {discharge$_1$, 2} (q2);
 \draw[->] (q2) to[out=20,in=160] node[above] {charge, 0} (q3);
 \draw[->] (q4) to[out=-110,in=-70,looseness=0.7] node[below] {discharge$_2$, 5} (q2);
 \draw[->] (q4) to[out=200,in=-20] node[below] {discharge$_1$, 2} (q3);
 \draw[->] (q3) to[out=20,in=160] node[above] {charge, 0} (q4);
 \end{tikzpicture}
}
	\caption{Soft constraint automata for energy management}\label{fig:myfigure}
\end{figure}
We denote the current state as a variable s. In the state $q_i$, the variable is of value $i$. 
We consider the set of constraint $C = \{charge, discharge_1, discharge_2\}$. The following formula $\Phi$ describes the behavior of the SCA of Figure 1.
\begin{align*}
\Phi =  & \quad charge \land s = 0 \land s'= 1 \land 0_\mathbb{W} \lor \\ 
& \quad charge \land s = 1 \land s'= 2 \land 0_\mathbb{W} \lor \\ 
& \quad charge \land s = 2 \land s'= 3 \land 0_\mathbb{W} \lor \\ 
& \quad charge \land s = 3 \land s'= 4 \land 0_\mathbb{W} \lor \\ & \quad discharge_{1} \land s = 4 \land s = 3 \land 2_\mathbb{W} \lor \\
& \quad discharge_{1} \land s = 3 \land s = 2 \land 2_\mathbb{W} \lor \\
& \quad discharge_{1} \land s = 2 \land s = 1 \land 2_\mathbb{W} \lor \\
& \quad discharge_{1} \land s = 1 \land s = 0 \land 2_\mathbb{W} \lor \\
& \quad discharge_{2} \land s = 4 \land s = 2 \land 5_\mathbb{W} \lor \\
& \quad discharge_{2} \land s = 3 \land s = 1 \land 5_\mathbb{W} \lor \\
& \quad discharge_{2} \land s = 2 \land s = 0 \land 5_\mathbb{W} \\
	= & \quad \bigvee_i \Phi_i
\end{align*}

The SCA of figure 1 uses the weighted c-semiring to express its preferences. In addition, we use the boolean c-semiring to represent the constraint as a c-semiring value. In order to show that the behavior of a SCA can be expressed by the formula $\Phi$, given a context $c$, we represent the formula with the following table :
$$
\begin{array}{l|cr}
& \mathbb{B} & \mathbb{W}\\
\hline
\Phi_1 & \quad \llbracket charge \land s = 0 \land s'= 1 \rrbracket_{c}  \quad & \quad 0_\mathbb{W} \\
\Phi_2 & \quad \llbracket charge \land s = 1 \land s'= 2 \rrbracket_{c} \quad & \quad 0_\mathbb{W} \\
\Phi_3 & \quad \llbracket charge \land s = 2 \land s'= 3 \rrbracket_{c} \quad & \quad 0_\mathbb{W} \\
\Phi_4 & \quad \llbracket charge \land s = 3 \land s'= 4 \rrbracket_{c} \quad & \quad 0_\mathbb{W} \\
\Phi_5 & \quad \llbracket discharge_{1} \land s = 3 \land s = 2 \rrbracket_{c} \quad & \quad 2_\mathbb{W} \\
\Phi_6 & \quad \llbracket discharge_{1} \land s = 2 \land s = 1 \rrbracket_{c} \quad & \quad 2_\mathbb{W} \\
\Phi_7 & \quad \llbracket discharge_{1} \land s = 1 \land s = 0 \rrbracket_{c} \quad & \quad 2_\mathbb{W} \\
\Phi_8 & \quad \llbracket discharge_{2} \land s = 4 \land s = 2 \rrbracket_{c} \quad & \quad 5_\mathbb{W} \\
\Phi_9 & \quad \llbracket discharge_{2} \land s = 3 \land s = 1 \rrbracket_{c} \quad & \quad 5_\mathbb{W} \\
\Phi_{10} & \quad \llbracket discharge_{2} \land s = 2 \land s = 0 \rrbracket_{c} \quad & \quad 5_\mathbb{W} \\
\end{array} 
$$
where $\llbracket \phi \rrbracket_{c} = \top$ if, given the context c and the free variable of $\phi$, the formula can be made true.
, the evaluation process first looks at the evaluation Starting by the initial context $c=\{s=4\}$, if 

\begin{figure}[H]
	\centering
	\resizebox{6cm}{!}{\input{fig/snapshot.tikz}}
	\caption{Soft constraint automaton modeling movement and snapshot}\label{fig:myfigure}
\end{figure}
Taking the variable m to represent the state of the SCA in figure 2, a corresponding formula modeling the behavior is :
\begin{align*}
\Psi =  & \quad move \land m = 0 \land m'= 0 \land 2_\mathbb{W} \lor \\ 
& \quad pass \land m = 0 \land m'= 0 \land 0_\mathbb{W} \lor \\ 
& \quad snapshot \land m = 0 \land m'= 1 \land 0_\mathbb{W} \lor \\ 
& \quad move \land m = 1 \land m' = 0 \land 0_\mathbb{W} \lor \\ 
& \quad pass \land m = 1 \land m' = 1 \land 2_\mathbb{W} \\
 = & \quad \bigvee_i \Psi_i \\
\end{align*}
\end{example}
$$
\begin{array}{l|cr}
& \mathbb{B} & \mathbb{W}\\
\hline
\Psi_1 & \quad \llbracket move \land m = 0 \land m'= 0 \rrbracket_{c}  \quad & \quad 2_\mathbb{W} \\
\Psi_2 & \quad \llbracket pass \land m = 0 \land m'= 0 \rrbracket_{c} \quad & \quad 0_\mathbb{W} \\
\Psi_3 & \quad \llbracket snapshot \land m = 0 \land m'= 1 \rrbracket_{c} \quad & \quad 0_\mathbb{W} \\
\Psi_4 & \quad \llbracket move \land m = 1 \land m'= 0 \rrbracket_{c} \quad & \quad 0_\mathbb{W} \\
\Psi_5 & \quad \llbracket pass \land m = 1 \land m' = 1 \rrbracket_{c} \quad & \quad 2_\mathbb{W} 
\end{array} 
$$

The composed SCA is defined as follow :
\begin{align*}
\Psi =  & \quad move \land m = 0 \land m'= 0 \land 2_\mathbb{W} \lor \\ 
& \quad pass \land m = 0 \land m'= 0 \land 0_\mathbb{W} \lor \\ 
& \quad snapshot \land m = 0 \land m'= 1 \land 0_\mathbb{W} \lor \\ 
& \quad move \land m = 1 \land m' = 0 \land 0_\mathbb{W} \lor \\ 
& \quad pass \land m = 1 \land m' = 1 \land 2_\mathbb{W} \\
= & \quad \bigvee_i \Psi_i \\
\end{align*}
