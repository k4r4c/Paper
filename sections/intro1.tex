\title{Working title}
%\author{Benjamin Lion\inst{1} \and Farhad Arbab\inst{2}}
%\institute{CWI, Amsterdam, Netherlands,\\
%	\email{I.Ekeland@princeton.edu},\\ WWW home page:
%	\texttt{http://users/\homedir iekeland/web/welcome.html}
%	\and
%	Universit\'{e} de Paris-Sud,
%	Laboratoire d'Analyse Num\'{e}rique, B\^{a}timent 425,\\
%	F-91405 Orsay Cedex, France}

\author{ }

\institute{ }

\maketitle

%The necessity of a coordination oriented formalization of complex systems to reason about properties under composition.
Soft Constraint Automata have been developed by Arbab \& Santini \cite{Arbab2013} and constitute a solid ground to design and reason about systems with preferences. In this paper, we present a complete procedure, from design of soft constraint systems, to code generation. The core concept of this procedure is the semantic preserving translation from SCA to a semiring logic, that enables new transformations. In the first part, we present definitions of Constraint Automata, semiring preferences and Soft Constraint Automata, and compare CA and SCA. In the second part, we focus on the new description of soft constraint systems using the definition of a semiring logic. 

In this paper, two main contributions are presented. First, the definition of a new composition framework for preferences in Soft Constraint Automata, by using the co-product. It allows us to extend the range of possible preferences composition. Second, the definition of a new concise model to represent in a unified way constraints and preferences into a semiring logic. In this framework, the logic is no more boolean, but semiring based. As a justification for those two points, we show along the paper that this new model is at least as expressive as Soft Constraint Automata, and can define composition operators as a quotient of the co-product.