\section{Preferences}

%\subsection*{Target}
%\begin{enumerate}
%	\item Constraint Semiring and intuition (choice and composition operator).
%	\item Properties of csemirings (induced order, some examples).
%	\item CoProduct of Constraint Semiring.
%	\item Quotient of Constraint Semiring.	
%\end{enumerate}

We want to get an intuition about how preferences interfere and describe a system's behavior. In the simple example described with west and east actions, the system's behavior can be described by sequences of actions taken by the system. Thus, the sequence $\langle west, east \rangle ^{\omega}$ is one infinite stream allowed by the automaton, representing the case where the robot is alternating between $west$ and $east$. Similarly, $\langle stay_{lon} \rangle ^{\omega}$ is also a possible behavior of the system, where the only action taken is $stay_{lon}$. Since constraint automata only tells whether a stream is part of the accepted behavior, we now search for expressing order among accepted behavior. Thus, $\Sigma^{\omega} \rightarrow 2$ becomes $\Sigma^{\omega} \rightarrow A$, where A is a suitable algebraic structure to order the streams.

There exist multiple way to define preferences. An approach used in \cite{bibid} uses multi valued systems. Lattices. Monoids. In \cite{bibid}, use of csemiring. The preference in choosing between different mathematical structure for preferences resides in certain properties internal to those structures. We want preferences to be \textbf{compositional} and (partially) \textbf{ordered}. We will justify in the second section why composition and order are necessary.

\subsection*{Constraint semiring and properties}
%The need for preferences can be justify by some simple example. If we look at the example of Fig. 1, on the state $q_W$, three different transitions are allowed : from $q_W$ to $q_E$ with constraint $east$, from $q_W$ to $q_W$ with constraint $west$ and from $q_W$ to $q_W$ with constraint $stay_{lon}$. If all of those three constraints are satisfiable, we will pick one non deterministically. But what if we want, in this case, to determine which one we prefer ? How could we encode that we prefer taking a transition instead of another one ?
Semirings constitutes a suitable mathematical structure to define the notion of preferences. A constraint semiring \cite{B04} induces an order relation among its element. Therefore, following the initial example on stream and behavior, each action of the stream is attached a preference in a c-semiring. The csemiring will serve as algebraic structure to order streams. We will look at the composition of set of streams with preferences, and the resulting stream order. 
Given two systems $A_1$ and $A_2$ with ordered stream and synchronization constraints ($A_1$ must synchronize along its run with $A_2$), what is the ordered stream generated by the parallel composition of $A_1$ and $A_2$ ? To answer this question, we must study the properties of the csemiring and order induced by the composition of different csemirings. 

A csemiring has an induced order defined given by the properties of its two operators : $+$ and $\times$. The $+$ operator can intuitively be seen as choosing the best value between two value. Thus, if $a,b \in E$, $a+b=a$ can be interpreted as $a$ "is preferred" to $b$. Alternatively, $\times$ is intuitively used to compose preferences. Then $a \times b$ is a new preference, which could be different from $a$ or $b$.

%As we want to compose ordered stream, we 
Since csemiring define an order among its element, and we have "intuition" behind its operators, we now want to look at the composition of different csemiring, and get intuition behind the order defined on the composed csemiring. Recall that we also want the composition to be ordered, thus being described by a csemiring.

On the search of a structure for our product of csemiring that :
\begin{list}{-}{}
	\item composed preferences of composite actions.
	\item induces an order on the composition.
	\item can later on be composed with a new csemiring.
\end{list}

The carrier of the semiring should contain any possible elements from the composite csemiring. Then, we call the free product such structure, being the cartesian product of all csemiring involved in the composition. It is possible to show that the direct product of two csemiring is itself a csemiring, where $+$ and $\times$ operator are interpreted as $+$ and $\times$ of the underlying csemiring. Thus, if $A_1$ compose with $A_2$, where preferences of $A_1$ are defined over a csemiring $E_1 \times E_2$ and preferences of $A_2$ are defined over $E_3$, we get $A = A_1 \times A_2$ with preferences defined over $E_1 \times E_2 \times E_3$.

At this point, remark that the definition of the composition is not without implication. Recall the csemiring induces an order among its element. By defining the product of csemiring as the cartesian product with interpretation of $\times$ and $+$ as underlying csemiring operators, we also define a certain order among the generated streams of the composed system. In this case, the induced order is the following : chose the best preference of the left csemiring, and chose the best preference of the right csemiring, and the resulting preference is the tuple consisting of those two preferences. 

\begin{example}
	The case where $\langle west, east \rangle ^{\omega}$ is a possible behavior of the system, and are attached some preferences : $\langle (west ,2), (east,5) \rangle ^{\omega}$ where $2$ and $5$ are value of the weighted csemiring. The stream described by $\langle (stay_{lon} ,3) \rangle ^{\omega}$ is also a possible behavior, but since $2<3$, the action $west$ is preferred to the action $stay_{lon}$, therefore the behavior described by the stream $\langle (west ,2), (east,5) \rangle ^{\omega}$ is preferred to the behavior described by the stream $\langle (stay_{lon} ,3) \rangle ^{\omega}$. However, taking one step further, in the next element of the stream, we now compare $\langle (east,5),(west ,2) \rangle ^{\omega}$ with $\langle (stay_{lon} ,3) \rangle ^{\omega}$. Since $3<5$, the stream order changed, and the first behavior (which was $\langle (east,5),(west ,2) \rangle ^{\omega}$) is no more preferred to the other behavior  $\langle (stay_{lon} ,3) \rangle ^{\omega}$. The system will now prefer to take the action $stay_{lon}$. Repeating this process, we finally get the stream $\langle (west,2),(stay_{lon},3) \rangle ^{\omega}$ as a preferred behavior.
\end{example}

\begin{example}
	Suppose a composition with a new behavior $\langle (snap,0.5) \rangle ^{\omega}$. By definition, the composed behavior are $\langle ((west,snap),(2,0.5)),((east,snap),(5,0.5)) \rangle ^{\omega}$ and $\langle ((stay_{lon},\emptyset),(3,\emptyset)) \rangle ^{\omega}$.
\end{example}




\begin{definition} (semiring)
	A semiring is a non empty set $E$ on which operations of addition and multiplication have been defined such that:
	\begin{list}{-}{ }
		\item $(E,+)$ is a commutative monoid with identity element 0.
		\item $(E,\times)$ is a monoid with identity element 1.
		\item Multiplication distributes over addition from either sides.
		\item $0 \times e = e \times 0 = 0$ for all $ e \in E$
	\end{list} 
	We note $\left\langle E, +, \times, \boldmath{1},\boldmath{0} \right\rangle$ to refer to this semiring.
\end{definition}

\begin{definition} (csemiring)
	A csemiring is a semiring  $\left\langle E, +, \times, \boldmath{1},\boldmath{0} \right\rangle$ with additional properties :
	\begin{list}{-}{ }
		\item + is idempotent. We use the notation $\sum(A)$ in prefix notation to describe the sum of all elements of a possibly infinite set $A \subset E$.
		\item $\times$ is commutative.
		\item $1 + e = e + 1 = 1$ for all $ e \in E$
	\end{list} 
	A csemiring admits a partial order $\leq_{E}$, defined as the smallest relation satisfying :
	$$
	\frac{e,e' \in E \quad e+e' = e}{e \leq_{E} e'}
	$$
\end{definition}

It is shown in [2] that $\leq$ satisfies the following properties:
\begin{list}{-}{ }
	\item $\leq$ is a partial order, with minimum 0 and maximum 1;
	\item $x + y$ is the least upper bound of x and y;
	\item $x \times y$ is a lower bound of x and y;
	\item (S, $\leq $) is a complete lattice (i.e., the greatest lower bound exists);
	\item + and $\times $ are monotone on $\leq$.
	\item if $\times $ is idempotent, then + distributes over $\times $, $x \times y$ is the greatest lower bound of $x$ and $y$, and (S, $\leq $) is a distributive lattice.
\end{list} 
%\begin{definition} (c-semiring)
%	A constraint semiring is a tuple $\left\langle \mathbb{E}, \bigvee, \otimes, \boldmath{1},\boldmath{0} \right\rangle$ where, for all $e\in E$, $E' \subset \mathbb{E}$, $\mathcal{E} \in 2^{\mathbb{E}}$, the following holds  :
%	\begin{list}{-}{ }
%		\item $\mathbb{E}$ is a set with $\boldmath{1},\boldmath{0} \in \mathbb{E}$
%		\item $\otimes : \mathbb{E} \times \mathbb{E} \rightarrow \mathbb{E}$ is a commutative and associative operator, with $\boldmath{0}_{\mathbb{E}} \otimes e = \boldmath{0}_{\mathbb{E}}$ and $\boldmath{1}_{\mathbb{E}} \otimes e = e$
%		\item $\bigvee : 2^{\mathbb{E}} \rightarrow \mathbb{E}$ , with $\bigvee{\mathbb{E}}=\boldmath{1} $, $\bigvee{\emptyset}=\boldmath{0}$ and $\bigvee_{E \in \mathcal{E}}{(\bigvee{E})}= \bigvee{(\bigcup \{E | E \in \mathcal{E}\})}$
%		\item $\otimes$ distributes over $\bigvee$ with $e \otimes \bigvee E = \bigvee \{ e \otimes e' | e' \in E \}$
%	\end{list} 
%	A c-semiring induces an order $\leq_{\mathbb{E}}$, defined as the smallest relation satisfying :
%	$$
%	\frac{e,e' \in \mathbb{E} \quad \bigvee\{e,e'\}=e}{e \leq_{\mathbb{E}} e'}
%	$$
%\end{definition}

Intuitively, the $\times$ operator behaves as a composition operator for preferences. The $\sum$ can be seen as the choice of the best preference over a set of preferences, where $1$ is the highest preference and $0$ is the lowest. We give some well known instances of c-semirings.

\begin{example}
Examples of c-semirings :
\begin{list}{-}{}
	\item The Boolean semiring $\mathbb{B}$:	$\left\langle \{\top,\bot\}, \lor, \land, \boldmath{1}_{\mathbb{B}}, \boldmath{0}_{\mathbb{B}} \right\rangle$ where $\boldmath{0}_{\mathbb{B}}=\bot$ and $\boldmath{1}_{\mathbb{B}}=\top$
	
	\item The Weighted semiring $\mathbb{W}$ :$\left\langle \mathbb{N}\cup\{\infty\}, \min, +, \boldmath{1}_{\mathbb{W}}, \boldmath{0}_{\mathbb{W}} \right\rangle$ where $\boldmath{0}_{\mathbb{W}}=\infty$ and $\boldmath{1}_{\mathbb{B}}=0$	
\end{list}

\paragraph{Application}
In the case of our trekker, he first uses boolean semiring to model his choice. If we $left$ and $right$ are propositional variable, we could model the trekker's choice by the following value :
$$ 	TrekChoice = left \lor right $$
Without any information, he can either chose left (make left true) or right.

Now that he has access to the distance, he can chose a weighted semiring instead, and model his choice by the following function :
$$
	TrekChoice = 	\left\{
	\begin{array}{rl}
	left \quad & \text{ if } l+_Wr= min(l,r) = l \\
	right \quad & \text{ otherwise }
	\end{array}
	\right.
$$
where $l$ is the time it takes on the left path, and $r$ on the right path.
%\paragraph{Example}
%\begin{list}{-}{}
%	\item The Boolean semiring $\mathbb{B}$:
%	$$\left\langle \{\top,\bot\}, \lor, \land, \boldmath{1}_{\mathbb{B}}, \boldmath{0}_{\mathbb{B}} \right\rangle$$ where $\boldmath{0}_{\mathbb{B}}=\bot$ and $\boldmath{1}_{\mathbb{B}}=\top$
%	Remarque : boolean semiring is a structure for valuation of first order logic sentences.
	
%	\item The Weighted semiring $\mathbb{W}$ :
%	$$\left\langle \mathbb{N}\cup\{\infty\}, \min, \max, \boldmath{1}_{\mathbb{W}}, \boldmath{0}_{\mathbb{W}} \right\rangle$$ where $\boldmath{0}_{\mathbb{W}}=\infty$ and $\boldmath{1}_{\mathbb{B}}=0$	
%\end{list}
\end{example}



Now that we expressed how to use csemiring to express preferences and choices, we want to be able to model multi criteria choices. Since a csemiring represent one projection of the choice (for instance smallest distance for a trekker), it would also be interesting to express several other dimensions (for instance hardness of the slope, beauty of the landscape, calm of the path, ..). We introduce in the next section the space where product of csemiring are defined : co-product.

\subsection*{Co-product of csemiring}

For this section, we assume $E=\langle E,+_E, \times_E,1,0 \rangle$ and $F=\langle F,+_F, \times_F,1,0 \rangle$ two c-semirings. 
We also assume $E \cap F = \mathbb{B}$, where $\mathbb{B}= \langle \{0,1\}, \times,+,1,0 \rangle$.

Note: 0 and 1 are same objects, shared by all c-semirings. For convenience, we write $e_i$ [resp. $f_i$] to refer to an element of the csemiring E [resp. F].

\begin{proposition}
	If $E \cap F \not = \mathbb{B}$, there exists an homomorphism h such that  $h(E) \cap F = \mathbb{B}$
\end{proposition}
\begin{proof}
	Define $h$ on $E$ such that h(e) = $\left\{
	\begin{array}{rl}
	(0,e) & \text{ if } e \not\in \mathbb{B} \\
	e \quad & \text{ otherwise }
	\end{array}
	\right.$. The map $h$ is homomorphic to E and $h(E) \cap F = \mathbb{B}$.
	
\end{proof}

\begin{definition} The tensor product $E\otimes_{\mathbb{B}}F$ is the tuple $\langle E \otimes_{\mathbb{B}} F, +, \times, 0, 1 \rangle $ where :
	\begin{list}{-}{}
	\item $E \subseteq E \otimes_{\mathbb{B}} F$ , $F \subseteq E \otimes_{\mathbb{B}} F$ and $E \otimes_{\mathbb{B}} F$ is closed under $+$ and $\times$.
	\item + is idempotent, associative and commutative.
	\item $\times$ is associative and commutative.
	\item $\times$ distributes over +.
	\item $\times$ and + are identified on E [resp. F] by $\times_E$ and $+_E$ [resp. $\times_F$ and $+_F$] .
	\end{list}
\end{definition}

\begin{lemma}
	All elements in $g \in E \otimes_{\mathbb{B}} F$ can be written as :
	$$
		g = \sum_i (e_i \times f_i)
	$$
	where $e_i \in E$ and $f_i \in F$
\end{lemma}
\begin{proof}
	Reasoning by induction on the term of $E \otimes_{\mathbb{B}} F$. For all elements $e\in E$ and $f\in F$, since $1 \in E \cap F$, we have:
	$$
	e \times 1 \in E \otimes_{\mathbb{B}} F \quad \text{ and } \quad	1 \times f \in E \otimes_{\mathbb{B}} F
	$$
	Suppose $g,h \in E\otimes_{\mathbb{B}} F$, where $g = \sum_i (e_i \times f_i)$ and $h = \sum_j (e_j \times f_j)$ having $e_i,e_j \in E$ and $f_i,f_j \in F$. We look at $g\times h \in E\otimes_{\mathbb{B}} F$ and $g + h \in E\otimes_{\mathbb{B}} F$:
	$$g \times h = \sum_i (e_i \times f_i) \times \sum_j (e_j \times f_j) = \sum_{i,j} (e_i \times f_i \times e_j \times f_j) = \sum_{i,j} (e_i \times e_j \times f_i \times f_j)$$
	where	$e_i \times e_j \in E  \text{ and }  f_i \times f_j \in E$,
	$$ g + h = \sum_i (e_i \times f_i) + \sum_j (e_j \times f_j) = \sum_{k} (e_k \times f_k)$$
	By induction, all terms of the tensor product $E \otimes_{\mathbb{B}} F$ can be represented as a sum over product of elements in $E$ and $F$.
\end{proof}

\begin{theorem}
	The tensor product $E \otimes_{\mathbb{B}} F$ is the co-product of E and F.
\end{theorem}
\begin{proof}
	We define two injection maps, $\iota_E : E \rightarrow E \otimes_{\mathbb{B}} F, e \mapsto e $ and $\iota_F : F \rightarrow E \otimes_{\mathbb{B}} F, f \mapsto f $. Given a csemiring G and $h_E : E \rightarrow G$ and $h_F : F \rightarrow G$ homomorphism, we want to prove the existence and uniqueness of $h : E \otimes_{\mathbb{B}} F \rightarrow G$ such that $h_E = h \circ \iota_E$ and $h_F = h \circ \iota_F$.
	
	We define $h : E \otimes_{\mathbb{B}} F \rightarrow G, \sum_i(e_i \times f_i) \mapsto \sum_i h_E(e_i)\times h_F(f_i)$. The map $h$ is well defined, since all elements of $E \otimes_{\mathbb{B}} F$ can be written as a sum over product of elements of E and F. Moreover, for all element $e\in E$, $h(\iota_E(e))=h_E(e) \times h_F(1) = h_E(e)$ and similarly for elements in F. The corresponding diagram commutes. To prove uniqueness of such map, let's define $h': E \otimes_{\mathbb{B}} F \rightarrow G, \sum_i(e_i \times f_i) \mapsto \sum_i h_E(e_i)\times h_F(f_i)$. Then, for all $i \in E \otimes_{\mathbb{B}} F$, $h'(i) = h(i)$, therefore $h=h'$.
\end{proof}

%There exists a natural embedding of csemiring E and F into the tensor product $E \otimes_{\mathbb{B}} F$ with the maps $i_E : E \rightarrow E \otimes_{\mathbb{B}} F, e \longmapsto e $ and $i_F : F \rightarrow E \otimes_{\mathbb{B}} F, f \longmapsto f  $.
%\begin{figure}[H]
%	\centering
%	\resizebox{8cm}{!}{ \begin{tikzpicture}[>=latex,shorten >=1pt,node distance=3cm,on grid,auto, node/.style={circle,draw,minimum size=25pt}, ]
 
 \node[draw=none] (q0) at (0pt,0pt) {$G$};
 \node[draw=none, right = of q0] (q1) {$F$};
 \node[draw=none, left = of q0] (q2) {$E$};
 \node[draw=none, above = of q0] (q3) {$E \otimes_{\mathbb{B}} F$};
 \draw[->] (q1) to node[above right] {$i_F$} (q3);
 \draw[->] (q2) to node[above left] {$i_E$} (q3);
 \draw[->] (q1) to node[above] {f} (q0);
 \draw[->] (q2) to node[above] {e} (q0);
 \draw[dashed,->] (q3) to node[right] {g} (q0);
 \end{tikzpicture}
}
%	\caption{Coproduct universal property}
%	\label{coprod}
%\end{figure}

%For this section, we assume $E=\langle E, \times,+,1,0 \rangle$ and $F=\langle F, \times,+,1,0 \rangle$ two c-semirings. 

\begin{definition} We define the disjoint union of two sets A and B while identifying elements of the intersection S:	
	$$A \cup_{S}B = (A \uplus B)/\sim, \quad x \sim y \Leftrightarrow x=y,\quad \text{where } y\in S$$
\end{definition}


Free(X) is the smallest set such that :
\begin{enumerate}
	\item $X \subseteq Free(X)$
	\item $x,y \in Free(X) \Rightarrow x \otimes y \in Free(X)$
	\item $x,y \in Free(X) \Rightarrow x + y \in Free(X)$
\end{enumerate}

We define ${\equiv} \subset Free(X)^2$ as the smallest congruence with regard to $\otimes,+$:
\begin{enumerate}
	\item $x \otimes y \equiv y \otimes x$
	\item $(x+x')\otimes y \equiv x \otimes y + x' \otimes y$
	\item $x+x \equiv x$
	\item $(x \otimes y) \otimes z \equiv x \otimes (y \otimes z)$
\end{enumerate}


\begin{definition} Given E,F csemiring and $\mathbb{B}$ the boolean csemiring, the tensor product  $E\otimes_{\mathbb{B}}F$ defined by the tuple $\langle Free(E \cup_{\mathbb{B}} F) /\equiv, +, \otimes, 0_{\mathbb{B}}, 1_{\mathbb{B}} \rangle $ is the co-product of E and F, where $0_{\mathbb{B}} \otimes a = 0_{\mathbb{B}}$, $1_{\mathbb{B}} + a = 1_{\mathbb{B}}$, $0_{\mathbb{B}} + a = a$ and $1_{\mathbb{B}} \otimes a = a$ for all $a \in E\otimes_{\mathbb{B}}F$
\end{definition}


%Since + is idempotent in a csemiring, the tensor product needs only be defined over a Boolean semiring. There exists a natural embedding of csemiring E and F into the tensor product $E \otimes_{\mathbb{B}} F$ with the maps $i_E : E \rightarrow E \otimes_{\mathbb{B}} F, e \longmapsto e $ and $i_F : F \rightarrow E \otimes_{\mathbb{B}} F, f \longmapsto f  $.
%\begin{figure}[H]
%	\centering
%	\resizebox{8cm}{!}{ \begin{tikzpicture}[>=latex,shorten >=1pt,node distance=3cm,on grid,auto, node/.style={circle,draw,minimum size=25pt}, ]
 
 \node[draw=none] (q0) at (0pt,0pt) {$G$};
 \node[draw=none, right = of q0] (q1) {$F$};
 \node[draw=none, left = of q0] (q2) {$E$};
 \node[draw=none, above = of q0] (q3) {$E \otimes_{\mathbb{B}} F$};
 \draw[->] (q1) to node[above right] {$i_F$} (q3);
 \draw[->] (q2) to node[above left] {$i_E$} (q3);
 \draw[->] (q1) to node[above] {f} (q0);
 \draw[->] (q2) to node[above] {e} (q0);
 \draw[dashed,->] (q3) to node[right] {g} (q0);
 \end{tikzpicture}
}
%	\caption{Coproduct universal property}
%	\label{coprod}
%\end{figure}

\subsection*{Quotients}

We are interested in quotients of $E \otimes_{\mathbb{B}} F$ as definition of different products of csemiring.

\begin{definition}(lexicographic) A quotient $h: E \otimes_{\mathbb{B}} F \rightarrow L$ is lexicographic if and only if, for all $e_1,e_2 \in E$ and $f_1,f_2 \in F$, there exists $i \in \{1,0\}$, such that we have:
	$$h(e_1 \times f_1 + e_2 \times f_2)= h(e_i \times f_i)$$
	whenever $e_1+e_2=e_i$ and $e_1 \not = e_2$; or $f_1 + f_2 = f_i$ and $e_1=e_2$
\end{definition}
\noindent
\begin{definition}
	(collapsing elements) We define the set of collapsing element of a semiring E by $$\mathcal{C}(E) = \{e \in E \quad | \quad \exists e_1 e_2 \in E, \quad e_1 \times e = e_2 \times e \land e_1 \not = e_2  \}$$
\end{definition}
In other words, it is possible to break a strict inequality after multiplication with a collapsing element. Collapsing element does not preserve strict inequality. Therefore, in the case of some product semiring (for instance lexicographic), distribution of $\times$ over + does not hold. When we come to define product, we should be aware of collapsing element in order to get back a csemiring. \\

Example : E, F csemirings. $e,e_1,e_2 \in E$ such that $e_1 \times e = e_2 \times e \land e_1 < e_2$ and $f_1,f_2 \in F$ with $f_2<f_1 $. We look at a possible element of the coproduct : $e_1 \times f_1 + e_2 \times f_2$. Assuming we have a quotient, and $h : E \sqcup F \rightarrow E \times_l F$ can project this term in a lexicographic product space, we would get $h(e_1 \times f_1 + e_2 \times f_2) = (e_2,f_2)$. By distributivity law, and given the homomorphic properties of h, we have :
\begin{align*}
	&h(e \times (e_1 \times f_1 + e_2 \times f_2)) = h(e \times e_1 \times f_1 + e \times e_2 \times f_2) = h(e \times e_2 \times f_1) \quad \\ \text{and} \\
	 &h(e) \times h(e_1 \times f_1 + e_2 \times f_2) = h(e) \times h( e_2 \times f_2) = h(e \times e_2 \times f_2) \\ \text{Thus} \\
	 & h(e \times (e_1 \times f_1 + e_2 \times f_2)) \not= h(e) \times h(e_1 \times f_1 + e_2 \times f_2)
\end{align*}

\begin{lemma}
	If E is cancellative, then there exists a lexicographic quotient $h : E \otimes_{\mathbb{B}} F \rightarrow L$
	%a relation $\equiv$ on $E \otimes_{\mathbb{B}} F$ such that for all $e_1,e_2 \in E$, $f_1,f_2 \in F$, there exists $i \in \{1,0\}$, such that we have :$$e_1 \times f_1 + e_2 \times f_2 \equiv e_i \times f_i$$ 	whenever $e_1+e_2=e_i$ and $e_1 \not = e_2$; or $f_1 + f_2 = f_i$ and $e_1=e_2$
\end{lemma}
%\begin{lemma}
%	The boolean csemiring is lexicographic
%\end{lemma}
%%\begin{definition}
%	C-semiring are R-algebra where R is the unital ring.
%\end{definition}

%The co-product of two csemiring E and F consists in the disjoint union of the elements of E, F and the tensor product $E \otimes_{\mathbb{B}} F$ over the Boolean semiring.
%\begin{definition}
%	The co-product $E \sqcup F$ of two c-semirings $E$ and $F$ is defined by 
%	$E \oplus F \oplus E \otimes_{\mathbb{B}} F$
%	modulo the equations $x + 1 = 1$ and $x \otimes 0 = 0$, for all x $\in E\sqcup F$.
%\end{definition}
% We note $\lfloor \cdot \rfloor : E \rightarrow E \sqcup F$ for the right injection and $\lceil \cdot \rceil : F \rightarrow E \sqcup F$ for the left injection.
\begin{theorem}
	Let E and F be csemiring. Then, given any csemiring G and maps $e : E \rightarrow G $ and $f : F \rightarrow G $, there exists a unique map $g : E \sqcup F \rightarrow G $ such that $g \circ \lfloor \cdot \rfloor  = e$ and $g \circ \lceil \cdot \rceil = f$
\end{theorem}

\begin{proof}
	By universal property of the co-product, there exists a unique map $g:E\sqcup F \rightarrow G$ such that the following diagram commutes:
	\begin{figure}[H]
		\centering
		\resizebox{8cm}{!}{ \begin{tikzpicture}[>=latex,shorten >=1pt,node distance=3cm,on grid,auto, node/.style={circle,draw,minimum size=25pt}, ]
 
 \node[draw=none] (q0) at (0pt,0pt) {$G$};
 \node[draw=none, right = of q0] (q1) {$F$};
 \node[draw=none, left = of q0] (q2) {$E$};
 \node[draw=none, above = of q0] (q3) {$E \otimes_{\mathbb{B}} F$};
 \draw[->] (q1) to node[above right] {$i_F$} (q3);
 \draw[->] (q2) to node[above left] {$i_E$} (q3);
 \draw[->] (q1) to node[above] {f} (q0);
 \draw[->] (q2) to node[above] {e} (q0);
 \draw[dashed,->] (q3) to node[right] {g} (q0);
 \end{tikzpicture}
}
		\caption{Coproduct universal property}
		\label{coprod}
	\end{figure}
\end{proof}

\begin{lemma}
	Given E a csemiring, there exists a map $g : E \sqcup E \rightarrow E$.
\end{lemma}
\begin{proof}
	property of the co-product.
\end{proof}

%\noindent
%\textbf{Properties of composition} 

\noindent
\textbf{collapsing element and coproduct} We define the set of collapsing element of a semiring E by $$\mathcal{C}(E) = \{e \in E \quad | \quad \exists e_1 e_2 \in E, \quad e_1 \times e = e_2 \times e \land e_1 < e_2  \}$$
In other words, it is possible to break a strict inequality after multiplication with a collapsing element. Collapsing element does not preserve strict inequality. Therefore, in the case of some product semiring (for instance lexicographic), distribution of $\times$ over + does not hold. When we come to define product, we should be aware of collapsing element in order to get back a csemiring. \\

Example : E, F csemirings. $e,e_1,e_2 \in E$ such that $e_1 \times e = e_2 \times e \land e_1 < e_2$ and $f_1,f_2 \in F$ with $f_2<f_1 $. We look at a possible element of the coproduct : $e_1 \otimes f_1 \oplus e_2 \otimes f_2$. Assuming we have a quotient, and $h : E \sqcup F \rightarrow E \times_l F$ can project this term in a lexicographic product space, we would get $h(e_1 \otimes f_1 \oplus e_2 \otimes f_2) = (e_2,f_2)$. However, now we would have the following :
$$
h(e \otimes e_1 \otimes f_1 \oplus e \otimes e_2 \otimes f_2) = (e \otimes_E e_2,f_1)  \not= (e \otimes_E e_2,f_2) = (e,1)\otimes_l (e_2,f_2) = h(e) \otimes_l h(e_1\otimes f1 \oplus e_2 \otimes f_2)
$$

%For the following definition, we restrict the carrier of a csemiring to its non-cancellative elements if we want the product to be a csemiring.

%We want to be able to define the following. Let E, F and C be three csemirings and $f: E \rightarrow C$, $g : F \rightarrow C $

%\begin{definition}
%	E and F csemirings, the co-product $E \sqcup F$ is defined by $E \oplus F \oplus E \otimes_{\mathbb{N}} F$ modulo $x \oplus 1 = 1 $ and $x \otimes 0 = 0 $.
%	\begin{list}{-}{ }
%		\item $x \oplus 1 = 1 $
%		\item $x \otimes 0 = 0 $
%	\end{list}
%\end{definition}

%\begin{definition}
%Given E and F csemirings, let $i_e : E \rightarrow E \otimes_{B} F$ and $i_f : F \rightarrow E \otimes_{B} F$ we define the co-product $E \otimes_{B} F$ where $B=\{0,1\}$ (since + is idempotent).
%\end{definition}


%\begin{definition}
%The co-product $\mathbb{E} \sqcup \mathbb{F}$ of two c-semirings $\mathbb{E}$ and $\mathbb{F}$ is defined by 
%$\mathbb{B} \oplus \mathbb{E} \oplus \mathbb{F} \oplus \mathbb{E} \otimes_{\mathbb{B}} \mathbb{F}$
%$\mathbb{E} \otimes_{\mathbb{B}} \mathbb{F}$
%, where $ \mathbb{B}$ is the boolean c-semiring.
%modulo the equations $x + 1 = 1$ and $x \otimes 0 = 0$, for all x $\in \mathbb{E}\sqcup \mathbb{F}$.
%\end{definition}

\begin{definition}Let $\mathbb{E}$ and $\mathbb{F}$ be two c-semirings. Their composition written $\mathbb{E} \sqcup \mathbb{F} = \left\langle \mathbb{E} \sqcup \mathbb{F}, +, \times, \boldmath{1},\boldmath{0} \right\rangle$ is defined as :
	\begin{list}{-}{}
		\item The carrier is given by the co-product $\mathbb{E} \sqcup \mathbb{F}$, and we write $\lfloor \cdot \rfloor : \mathbb{E} \rightarrow \mathbb{E} \sqcup \mathbb{F}$ and $\lceil \cdot \rceil : \mathbb{F} \rightarrow \mathbb{E} \sqcup \mathbb{F}$ the left and right injections.
		\item The choice operator $+ : 2^{\mathbb{E} \sqcup \mathbb{F}} \rightarrow \mathbb{E} \sqcup \mathbb{F}$, where, given $e_1,e_2 \in \mathbb{E}$ and $s_1,s_2 \in \mathbb{F}$: : 
		\begin{align*}
			& \lceil e_1 \rceil + \lceil e_2 \rceil = \lceil e_1 + e_2 \rceil \\
			& \lfloor s_1 \rfloor + \lfloor s_2 \rfloor = \lfloor s_1 + s_2 \rfloor  %\\
			%			& \lfloor s_1 \rfloor \times \lceil e_1 \rceil + \lfloor s_2 \rfloor \times \lceil e_2 \rceil = \lfloor s_1 + s_2 \rfloor \times \lceil e_1 +e_2 \rceil
		\end{align*}
		\item The composition operator $\times : \mathbb{E} \sqcup \mathbb{F} \times \mathbb{E} \sqcup \mathbb{F} \rightarrow \mathbb{E} \sqcup \mathbb{F}$, where, given $e_1,e_2 \in \mathbb{E}$ and $s_1,s_2 \in \mathbb{F}$:
		\begin{align*}
			& \lceil e_1 \rceil \times \lceil e_2 \rceil = \lceil e_1 \times e_2 \rceil \\
			& \lfloor s_1 \rfloor \times \lfloor s_2 \rfloor = \lfloor s_1 \times s_2 \rfloor 
		\end{align*}
		\item $\textbf{0}_{\mathbb{E} \sqcup \mathbb{F}}=\lceil\textbf{0}_{\mathbb{E}}\rceil=\lfloor\textbf{0}_{\mathbb{F}}\rfloor$ and $\textbf{1}_{\mathbb{E} \sqcup \mathbb{F}}=\lceil\textbf{1}_{\mathbb{E}}\rceil=\lfloor\textbf{1}_{\mathbb{F}}\rfloor$
	\end{list}
\end{definition}


%\begin{theorem}
	Let $\mathbb{E}$ and $\mathbb{G}$ two c-semirings such that $e : \mathbb{E} \rightarrow \mathbb{G} $ is an homomorphism between $ \mathbb{E} $ and $ \mathbb{G} $. There exists a unique morphism $g :  \mathbb{E} \sqcup \mathbb{E} \rightarrow \mathbb{G}$ such that
\end{theorem}

\begin{proof} 
	We define $i_1 = \mathbb{E} \rightarrow \mathbb{E} \sqcup \mathbb{E} $ the right injection and $i_2 = \mathbb{E} \rightarrow \mathbb{E} \sqcup \mathbb{E} $ as the left injection.
	%	For this proof, we show that $\mathbb{E} \sqcup \mathbb{E} = \left\langle \mathbb{E} \sqcup \mathbb{E}, +_{\mathbb{E} \sqcup \mathbb{E}}, \times_{\mathbb{E} \sqcup \mathbb{E}}, \boldmath{1}_{\mathbb{E} \sqcup \mathbb{E}},\boldmath{0}_{\mathbb{E} \sqcup \mathbb{E}} \right\rangle =  \left\langle \mathbb{E}, +_{\mathbb{E}}, \times_{\mathbb{E}}, \boldmath{1}_{\mathbb{E}},\boldmath{0}_{\mathbb{E}} \right\rangle$. We note $\lceil \cdot \rceil : \mathbb{E} \rightarrow \mathbb{E}\sqcup\mathbb{E}$ and by definition we have $ \boldmath{1}_{\mathbb{E}\sqcup\mathbb{E}} =  \boldmath{1}_{\mathbb{E}} $ and  $ \boldmath{0}_{\mathbb{E}\sqcup\mathbb{E}} =  \boldmath{0}_{\mathbb{E}} $. The carrier is written $\mathbb{B} \oplus \mathbb{E} \oplus \mathbb{E} \oplus \mathbb{E} \otimes_{\mathbb{B}}\mathbb{E}$
\end{proof}



%We are interested in quotients of $E \otimes_{\mathbb{B}} F$ as definition of different products of csemiring.

\begin{definition}(lexicographic) A quotient $h: E \otimes_{\mathbb{B}} F \rightarrow L$ is lexicographic if and only if, for all $e_1,e_2 \in E$ and $f_1,f_2 \in F$, there exists $i \in \{1,0\}$, such that we have:
	$$h(e_1 \times f_1 + e_2 \times f_2)= h(e_i \times f_i)$$
	whenever $e_1+e_2=e_i$ and $e_1 \not = e_2$; or $f_1 + f_2 = f_i$ and $e_1=e_2$
\end{definition}
\noindent
\begin{definition}
	(collapsing elements) We define the set of collapsing element of a semiring E by $$\mathcal{C}(E) = \{e \in E \quad | \quad \exists e_1 e_2 \in E, \quad e_1 \times e = e_2 \times e \land e_1 \not = e_2  \}$$
\end{definition}
In other words, it is possible to break a strict inequality after multiplication with a collapsing element. Collapsing element does not preserve strict inequality. Therefore, in the case of some product semiring (for instance lexicographic), distribution of $\times$ over + does not hold. When we come to define product, we should be aware of collapsing element in order to get back a csemiring. \\

Example : E, F csemirings. $e,e_1,e_2 \in E$ such that $e_1 \times e = e_2 \times e \land e_1 < e_2$ and $f_1,f_2 \in F$ with $f_2<f_1 $. We look at a possible element of the coproduct : $e_1 \times f_1 + e_2 \times f_2$. Assuming we have a quotient, and $h : E \sqcup F \rightarrow E \times_l F$ can project this term in a lexicographic product space, we would get $h(e_1 \times f_1 + e_2 \times f_2) = (e_2,f_2)$. By distributivity law, and given the homomorphic properties of h, we have :
\begin{align*}
	&h(e \times (e_1 \times f_1 + e_2 \times f_2)) = h(e \times e_1 \times f_1 + e \times e_2 \times f_2) = h(e \times e_2 \times f_1) \quad \\ \text{and} \\
	 &h(e) \times h(e_1 \times f_1 + e_2 \times f_2) = h(e) \times h( e_2 \times f_2) = h(e \times e_2 \times f_2) \\ \text{Thus} \\
	 & h(e \times (e_1 \times f_1 + e_2 \times f_2)) \not= h(e) \times h(e_1 \times f_1 + e_2 \times f_2)
\end{align*}

\begin{lemma}
	If E is cancellative, then there exists a lexicographic quotient $h : E \otimes_{\mathbb{B}} F \rightarrow L$
	%a relation $\equiv$ on $E \otimes_{\mathbb{B}} F$ such that for all $e_1,e_2 \in E$, $f_1,f_2 \in F$, there exists $i \in \{1,0\}$, such that we have :$$e_1 \times f_1 + e_2 \times f_2 \equiv e_i \times f_i$$ 	whenever $e_1+e_2=e_i$ and $e_1 \not = e_2$; or $f_1 + f_2 = f_i$ and $e_1=e_2$
\end{lemma}
%\begin{lemma}
%	The boolean csemiring is lexicographic
%\end{lemma}

%\begin{proposition}
	Given $\mathbb{E}$ and $\mathbb{F}$ c-semiring, their composition defined by the co-product $\mathbb{E} \sqcup \mathbb{F}$ is a c-semiring.
\end{proposition}
\begin{proof}Properties of + and $\times$ inherited from underlying semiring.
\end{proof}

\begin{proposition} Given $\mathbb{E}$ and $\mathbb{F}$ c-semiring, for all element $s \in \mathbb{E} \sqcup \mathbb{F}$, there exist $s_1,s_3 \in \mathbb{E}$, $s_2,s_4 \in \mathbb{F}$ such that
	$$
	s = \lceil s_1 \rceil + \lfloor s_2 \rfloor + \lceil s_3 \rceil \times \lfloor s_4 \rfloor
	$$
	is a normal form.
\end{proposition}
