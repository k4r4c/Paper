\title{Just-in-time composition}
\author{Benjamin Lion\inst{1} \and Hans-Dieter Hiep\inst{1} \and Kasper Dokter\inst{1} \and Farhad Arbab\inst{1} \and Carolyn Talcott\inst{2}}
\institute{CWI, Amsterdam, Netherlands,\\
	\and
	SRI}

%\author{ }

%\institute{}

\maketitle

\begin{abstract}

%	The observable behavior of a system can be described as a set of streams of observation. From a designer point of view, the objective is to produce
	
	
	
	% A system can be described by its set of streams of observable behavior. From a designer point of view, the objective is to model the intended behavior as part of the set of observable stream generated by the model. 
	Soft constraint automata generalize constraint automata in the sense that they define a partial order on the set of their observable behaviors. The partial order results from the composition of c-semiring values. In existing work, the composite partial order requires complete knowledge of the domain before composition. Soft constraint automata constitute a very expressive model for open distributed systems. However, open systems require support for composition with possibly unknown domains. In this paper, we present a novel composition of soft constraint automata that allows run-time interpretation of free c-semiring expressions, to determine a partial order. Our technique is at least as expressive as the known composition, and we show by examples, how by delaying concrete composition until run-time, our new composition extends the applicability of c-semiring composition in modeling of open distributed systems.
	
%	A system is a set of acceptable behavior, where in our case behaviors are represented as streams of data. In the case of open system, the set of acceptable is conditioned on the environment behavior. 
	
%	Behavior can be represented as stream of data, and a system as a set of accepting streams. When
	
	
%	Soft constraint automata are constraint automata with, at every transition, a value from a c-semiring.	These values model preferences that influence otherwise non-deterministic choices.
%	Currently, composition of soft constraint automata is defined over a single global c-semiring. However, it is not always clear what global c-semiring is suitable. In this paper, we propose a composition of soft constraint automata that take values from different c-semirings. We prove that the current composition and our proposed composition are equally expressive.
	

	%Open systems are difficult to model. 
	%They need to integrate a mechanism to cope with uncertainty. Soft constraint automata represent
	
%	Soft constraint automata extend constraint automata by associating, for each transition, a preference value from a csemiring. Preferences model concerns, and multiple concern can be involved in a single choice. We present a new framework to compose preferences of difference csemiring using a co-product. Since constraint automata have an interpretation as a boolean formula, we then investigate the interpretation of soft constraint automata as a semiring formula. 
	
	%The definition of a semiring logic requires a proper definition of product space of semiring.
	
%	Constraint automata have an logical interpretation that reduces state transition explosion.
%	Soft constraint automata, logic with csemiring, decoupling product from quotient, new composition operator.
\end{abstract}


In cyber-physical systems, one main objective is to cope with unpredictable environments of agents.
We model agents and their immediate observable environment as being an autonomous system.
Cyber-physical systems and autonomous systems are instances of \emph{open distributed systems}.
The techniques we present in this paper are applicable in general.
We focus our running example on a cyber-physical application to animate the mind of the reader.

To set the stage, let us consider an example scenario of mobile agents.
An agent is a device with communication mechanisms,
and other sensing and acting peripherals,
that has an unpredictable environment:
the agent's immediate environment can change by the entering or leaving of other nearby agents,
and in general, sensors measure changes of the environment as induced by external stimuli.
Agents may act in their environment\textemdash for example,
moving around by driving motors or by other propulsion.
The resulting behavior leads in turn to a change in environment.

The open-endedness of agents' environments poses a challenge:
we cannot \emph{a priori} know all possible configurations of an autonomous system,
since each configuration can lead to a potentially infinite number of possible behaviors.
How do we design and implement agents that still have reasonably acceptable behavior
in such unpredictable environments?
