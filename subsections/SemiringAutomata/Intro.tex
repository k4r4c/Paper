Defining constraint predicate and c-semiring are usually orthogonal problems. In Soft Constraint Automata, boolean constraints and c-semiring are separately defined and assigned to each transition. Conceptually, we could find some justifications for this separation : the system should first look at the enable transitions (which boolean constraint is true) and chose the best transition (regarding semiring value). Enabling and ordering are two different concerns. The problem is more practical, and arises during composition. Due to the separation between constraint and c-semiring, the composition operator must differentiate both concerns. In this section, we propose a unified formal model to express both constraint and c-semiring as a soft constraint predicate. Intuitively, a soft constraint predicate is the composition of a c-semiring value from boolean semiring (e.g. the constraint), composed with another c-semiring value (e.g. the semiring value). A new composition operator is lately presented.

%An autonomous system has the particularity of evolving in autonomy inside an environment. Decisions must be periodically made and one way to formally represent the behavior is to use Constraint Automata (CA). States in CA are a configuration of the system (variable assignment). Transitions contain constraint on variable assignment (logical formula), and a transition is allowed if the constraint is satisfied. However, CA are non deterministic automata : from a state, if two transitions are allowed, one transition is chosen non deterministically. Determinising CA is as costly as determinising Buchï automaton : double exponential.
%Constraint Automata is a suitable model to express system in interaction. Each models 

%Let's think about any system and try to formally define it. 

%We start by reminding some notion related to Constraint Automata (CA). An intuitive view of how a CA could be executed is as follow. A Constraint Automata involves a set of variable $V$. Each state of the Constraint Automata is a particular set of variable's assignment. The assignment needs not to be complete (ie, some variable could still be free). From the state, a set of outgoing transitions is defined. A transition involves a constraint, represented by a boolean predicate over a subset of the system's variables. If all the variables involved in the predicate are defined, the formula is evaluated to $true$ or $false$. If some variables are not defined, the assignment is made non deterministically, and the predicate is evaluated.
