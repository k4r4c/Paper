\subsection{Soft Constraint Automata}

\begin{definition}
	Soft Constraint Automata (SCA) is a tuple $\left\langle Q, \rightarrow, C, \mathbb{S}, q_{0}, \right\rangle$ where : 
	\begin{list}{-}{ }
		\item $Q$ is a set of state and $q_0\in$ Q is the initial state.
		\item C is a set of constraints.
		\item $\rightarrow \subseteq Q \times C \times \mathbb{S} \times Q$ is a finite relation called transition relation. 
		%A transition is denoted by $\left\langle q_i, c, s, p_i \right\rangle \in \rightarrow$
	\end{list}
\end{definition}

\begin{example}
	The automaton of Fig.2 represents the same 
	\begin{figure}[H]
		\centering
		\resizebox{8cm}{!}{ \begin{tikzpicture}[>=latex,shorten >=1pt,node distance=3cm,on grid,auto, node/.style={circle,draw,minimum size=25pt}, ]

 \node[state] (q0) at (-40pt,0pt) {$q_W$};
 \node[state, right = of q0] (q1) {$q_E$};
 \draw[<-,text=white] (q0) -- node[] {} ++(0,1);
 \draw[->] (q1) to[out=200,in=-20] node[below] {west , 5} (q0);
 \draw[->] (q0) to[out=20,in=160] node[above] {east , 5} (q1);
 \draw[->] (q1) to[out=20,in=-20,looseness=8] node[right, align=left] {east , 0\\stay$_{lon}$ , 5} (q1);
 \draw[->] (q0) to[out=160,in=-160,looseness=8] node[left, align=left] {west , 0\\ stay$_{lon}$ , 5} (q0);
 \end{tikzpicture}
}
		\caption{Soft constraint automata for moving}
		\label{moveSCA}
	\end{figure}
\end{example}


\begin{definition}
	Given $A_1=\left\langle Q_1, \rightarrow_1, C_1,\mathbb{S}, q_{01} \right\rangle$ and $A_2=\left\langle Q_2, \rightarrow_2, C_2,\mathbb{S}, q_{02} \right\rangle$ two SCA, we define the product $ A_1 \times A_2 = \left \langle Q, \rightarrow, C,\mathbb{S}, (q_{01},q_{02}) \right \rangle$ where : 
	\begin{list}{-}{ }
		\item $Q= Q_1 \times Q_2 $ is a set of state and $(q_{01},q_{02})\in$ Q is the initial state.
		\item The transition relation $\rightarrow$ is the smallest relation satisfying
		$$
		\frac{q_{01} \xrightarrow{c_1,e_1}_1 q_{01}' ,\quad q_{02} \xrightarrow{c_2,e_2}_2 q_{02}' }{(q_{01},q_{02}) \xrightarrow{c_1 \land c_2,e_1 \otimes e_2}(q_{01}',q_{02}')}
		$$
		%A transition is denoted by $\left\langle q_i, c, s, p_i \right\rangle \in \rightarrow$
	\end{list}
\end{definition}
