\begin{example}
Examples of c-semirings :
\begin{list}{-}{}
	\item The Boolean semiring $\mathbb{B}$:	$\left\langle \{\top,\bot\}, \lor, \land, \boldmath{1}_{\mathbb{B}}, \boldmath{0}_{\mathbb{B}} \right\rangle$ where $\boldmath{0}_{\mathbb{B}}=\bot$ and $\boldmath{1}_{\mathbb{B}}=\top$
	
	\item The Weighted semiring $\mathbb{W}$ :$\left\langle \mathbb{N}\cup\{\infty\}, \min, +, \boldmath{1}_{\mathbb{W}}, \boldmath{0}_{\mathbb{W}} \right\rangle$ where $\boldmath{0}_{\mathbb{W}}=\infty$ and $\boldmath{1}_{\mathbb{B}}=0$	
\end{list}

\paragraph{Application}
In the case of our trekker, he first uses boolean semiring to model his choice. If we $left$ and $right$ are propositional variable, we could model the trekker's choice by the following value :
$$ 	TrekChoice = left \lor right $$
Without any information, he can either chose left (make left true) or right.

Now that he has access to the distance, he can chose a weighted semiring instead, and model his choice by the following function :
$$
	TrekChoice = 	\left\{
	\begin{array}{rl}
	left \quad & \text{ if } l+_Wr= min(l,r) = l \\
	right \quad & \text{ otherwise }
	\end{array}
	\right.
$$
where $l$ is the time it takes on the left path, and $r$ on the right path.
%\paragraph{Example}
%\begin{list}{-}{}
%	\item The Boolean semiring $\mathbb{B}$:
%	$$\left\langle \{\top,\bot\}, \lor, \land, \boldmath{1}_{\mathbb{B}}, \boldmath{0}_{\mathbb{B}} \right\rangle$$ where $\boldmath{0}_{\mathbb{B}}=\bot$ and $\boldmath{1}_{\mathbb{B}}=\top$
%	Remarque : boolean semiring is a structure for valuation of first order logic sentences.
	
%	\item The Weighted semiring $\mathbb{W}$ :
%	$$\left\langle \mathbb{N}\cup\{\infty\}, \min, \max, \boldmath{1}_{\mathbb{W}}, \boldmath{0}_{\mathbb{W}} \right\rangle$$ where $\boldmath{0}_{\mathbb{W}}=\infty$ and $\boldmath{1}_{\mathbb{B}}=0$	
%\end{list}
\end{example}

