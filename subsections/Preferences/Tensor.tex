For this section, we assume $E=\langle E,+_E, \times_E,1,0 \rangle$ and $F=\langle F,+_F, \times_F,1,0 \rangle$ two c-semirings. 
We also assume $E \cap F = \mathbb{B}$, where $\mathbb{B}= \langle \{0,1\}, \times,+,1,0 \rangle$.

Note: 0 and 1 are same objects, shared by all c-semirings. For convenience, we write $e_i$ [resp. $f_i$] to refer to an element of the csemiring E [resp. F].

\begin{proposition}
	If $E \cap F \not = \mathbb{B}$, there exists an homomorphism h such that  $h(E) \cap F = \mathbb{B}$
\end{proposition}
\begin{proof}
	Define $h$ on $E$ such that h(e) = $\left\{
	\begin{array}{rl}
	(0,e) & \text{ if } e \not\in \mathbb{B} \\
	e \quad & \text{ otherwise }
	\end{array}
	\right.$. The map $h$ is homomorphic to E and $h(E) \cap F = \mathbb{B}$.
	
\end{proof}

\begin{definition} The tensor product $E\otimes_{\mathbb{B}}F$ is the tuple $\langle E \otimes_{\mathbb{B}} F, +, \times, 0, 1 \rangle $ where :
	\begin{list}{-}{}
	\item $E \subseteq E \otimes_{\mathbb{B}} F$ , $F \subseteq E \otimes_{\mathbb{B}} F$ and $E \otimes_{\mathbb{B}} F$ is closed under $+$ and $\times$.
	\item + is idempotent, associative and commutative.
	\item $\times$ is associative and commutative.
	\item $\times$ distributes over +.
	\item $\times$ and + are identified on E [resp. F] by $\times_E$ and $+_E$ [resp. $\times_F$ and $+_F$] .
	\end{list}
\end{definition}

\begin{lemma}
	All elements in $g \in E \otimes_{\mathbb{B}} F$ can be written as :
	$$
		g = \sum_i (e_i \times f_i)
	$$
	where $e_i \in E$ and $f_i \in F$
\end{lemma}
\begin{proof}
	Reasoning by induction on the term of $E \otimes_{\mathbb{B}} F$. For all elements $e\in E$ and $f\in F$, since $1 \in E \cap F$, we have:
	$$
	e \times 1 \in E \otimes_{\mathbb{B}} F \quad \text{ and } \quad	1 \times f \in E \otimes_{\mathbb{B}} F
	$$
	Suppose $g,h \in E\otimes_{\mathbb{B}} F$, where $g = \sum_i (e_i \times f_i)$ and $h = \sum_j (e_j \times f_j)$ having $e_i,e_j \in E$ and $f_i,f_j \in F$. We look at $g\times h \in E\otimes_{\mathbb{B}} F$ and $g + h \in E\otimes_{\mathbb{B}} F$:
	$$g \times h = \sum_i (e_i \times f_i) \times \sum_j (e_j \times f_j) = \sum_{i,j} (e_i \times f_i \times e_j \times f_j) = \sum_{i,j} (e_i \times e_j \times f_i \times f_j)$$
	where	$e_i \times e_j \in E  \text{ and }  f_i \times f_j \in E$,
	$$ g + h = \sum_i (e_i \times f_i) + \sum_j (e_j \times f_j) = \sum_{k} (e_k \times f_k)$$
	By induction, all terms of the tensor product $E \otimes_{\mathbb{B}} F$ can be represented as a sum over product of elements in $E$ and $F$.
\end{proof}

\begin{theorem}
	The tensor product $E \otimes_{\mathbb{B}} F$ is the co-product of E and F.
\end{theorem}
\begin{proof}
	We define two injection maps, $\iota_E : E \rightarrow E \otimes_{\mathbb{B}} F, e \mapsto e $ and $\iota_F : F \rightarrow E \otimes_{\mathbb{B}} F, f \mapsto f $. Given a csemiring G and $h_E : E \rightarrow G$ and $h_F : F \rightarrow G$ homomorphism, we want to prove the existence and uniqueness of $h : E \otimes_{\mathbb{B}} F \rightarrow G$ such that $h_E = h \circ \iota_E$ and $h_F = h \circ \iota_F$.
	
	We define $h : E \otimes_{\mathbb{B}} F \rightarrow G, \sum_i(e_i \times f_i) \mapsto \sum_i h_E(e_i)\times h_F(f_i)$. The map $h$ is well defined, since all elements of $E \otimes_{\mathbb{B}} F$ can be written as a sum over product of elements of E and F. Moreover, for all element $e\in E$, $h(\iota_E(e))=h_E(e) \times h_F(1) = h_E(e)$ and similarly for elements in F. The corresponding diagram commutes. To prove uniqueness of such map, let's define $h': E \otimes_{\mathbb{B}} F \rightarrow G, \sum_i(e_i \times f_i) \mapsto \sum_i h_E(e_i)\times h_F(f_i)$. Then, for all $i \in E \otimes_{\mathbb{B}} F$, $h'(i) = h(i)$, therefore $h=h'$.
\end{proof}

%There exists a natural embedding of csemiring E and F into the tensor product $E \otimes_{\mathbb{B}} F$ with the maps $i_E : E \rightarrow E \otimes_{\mathbb{B}} F, e \longmapsto e $ and $i_F : F \rightarrow E \otimes_{\mathbb{B}} F, f \longmapsto f  $.
%\begin{figure}[H]
%	\centering
%	\resizebox{8cm}{!}{ \begin{tikzpicture}[>=latex,shorten >=1pt,node distance=3cm,on grid,auto, node/.style={circle,draw,minimum size=25pt}, ]
 
 \node[draw=none] (q0) at (0pt,0pt) {$G$};
 \node[draw=none, right = of q0] (q1) {$F$};
 \node[draw=none, left = of q0] (q2) {$E$};
 \node[draw=none, above = of q0] (q3) {$E \otimes_{\mathbb{B}} F$};
 \draw[->] (q1) to node[above right] {$i_F$} (q3);
 \draw[->] (q2) to node[above left] {$i_E$} (q3);
 \draw[->] (q1) to node[above] {f} (q0);
 \draw[->] (q2) to node[above] {e} (q0);
 \draw[dashed,->] (q3) to node[right] {g} (q0);
 \end{tikzpicture}
}
%	\caption{Coproduct universal property}
%	\label{coprod}
%\end{figure}