%\begin{definition}
%	C-semiring are R-algebra where R is the unital ring.
%\end{definition}

%The co-product of two csemiring E and F consists in the disjoint union of the elements of E, F and the tensor product $E \otimes_{\mathbb{B}} F$ over the Boolean semiring.
%\begin{definition}
%	The co-product $E \sqcup F$ of two c-semirings $E$ and $F$ is defined by 
%	$E \oplus F \oplus E \otimes_{\mathbb{B}} F$
%	modulo the equations $x + 1 = 1$ and $x \otimes 0 = 0$, for all x $\in E\sqcup F$.
%\end{definition}
% We note $\lfloor \cdot \rfloor : E \rightarrow E \sqcup F$ for the right injection and $\lceil \cdot \rceil : F \rightarrow E \sqcup F$ for the left injection.
\begin{theorem}
	Let E and F be csemiring. Then, given any csemiring G and maps $e : E \rightarrow G $ and $f : F \rightarrow G $, there exists a unique map $g : E \sqcup F \rightarrow G $ such that $g \circ \lfloor \cdot \rfloor  = e$ and $g \circ \lceil \cdot \rceil = f$
\end{theorem}

\begin{proof}
	By universal property of the co-product, there exists a unique map $g:E\sqcup F \rightarrow G$ such that the following diagram commutes:
	\begin{figure}[H]
		\centering
		\resizebox{8cm}{!}{ \begin{tikzpicture}[>=latex,shorten >=1pt,node distance=3cm,on grid,auto, node/.style={circle,draw,minimum size=25pt}, ]
 
 \node[draw=none] (q0) at (0pt,0pt) {$G$};
 \node[draw=none, right = of q0] (q1) {$F$};
 \node[draw=none, left = of q0] (q2) {$E$};
 \node[draw=none, above = of q0] (q3) {$E \otimes_{\mathbb{B}} F$};
 \draw[->] (q1) to node[above right] {$i_F$} (q3);
 \draw[->] (q2) to node[above left] {$i_E$} (q3);
 \draw[->] (q1) to node[above] {f} (q0);
 \draw[->] (q2) to node[above] {e} (q0);
 \draw[dashed,->] (q3) to node[right] {g} (q0);
 \end{tikzpicture}
}
		\caption{Coproduct universal property}
		\label{coprod}
	\end{figure}
\end{proof}

\begin{lemma}
	Given E a csemiring, there exists a map $g : E \sqcup E \rightarrow E$.
\end{lemma}
\begin{proof}
	property of the co-product.
\end{proof}

%\noindent
%\textbf{Properties of composition} 

\noindent
\textbf{collapsing element and coproduct} We define the set of collapsing element of a semiring E by $$\mathcal{C}(E) = \{e \in E \quad | \quad \exists e_1 e_2 \in E, \quad e_1 \times e = e_2 \times e \land e_1 < e_2  \}$$
In other words, it is possible to break a strict inequality after multiplication with a collapsing element. Collapsing element does not preserve strict inequality. Therefore, in the case of some product semiring (for instance lexicographic), distribution of $\times$ over + does not hold. When we come to define product, we should be aware of collapsing element in order to get back a csemiring. \\

Example : E, F csemirings. $e,e_1,e_2 \in E$ such that $e_1 \times e = e_2 \times e \land e_1 < e_2$ and $f_1,f_2 \in F$ with $f_2<f_1 $. We look at a possible element of the coproduct : $e_1 \otimes f_1 \oplus e_2 \otimes f_2$. Assuming we have a quotient, and $h : E \sqcup F \rightarrow E \times_l F$ can project this term in a lexicographic product space, we would get $h(e_1 \otimes f_1 \oplus e_2 \otimes f_2) = (e_2,f_2)$. However, now we would have the following :
$$
h(e \otimes e_1 \otimes f_1 \oplus e \otimes e_2 \otimes f_2) = (e \otimes_E e_2,f_1)  \not= (e \otimes_E e_2,f_2) = (e,1)\otimes_l (e_2,f_2) = h(e) \otimes_l h(e_1\otimes f1 \oplus e_2 \otimes f_2)
$$

%For the following definition, we restrict the carrier of a csemiring to its non-cancellative elements if we want the product to be a csemiring.

%We want to be able to define the following. Let E, F and C be three csemirings and $f: E \rightarrow C$, $g : F \rightarrow C $

%\begin{definition}
%	E and F csemirings, the co-product $E \sqcup F$ is defined by $E \oplus F \oplus E \otimes_{\mathbb{N}} F$ modulo $x \oplus 1 = 1 $ and $x \otimes 0 = 0 $.
%	\begin{list}{-}{ }
%		\item $x \oplus 1 = 1 $
%		\item $x \otimes 0 = 0 $
%	\end{list}
%\end{definition}

%\begin{definition}
%Given E and F csemirings, let $i_e : E \rightarrow E \otimes_{B} F$ and $i_f : F \rightarrow E \otimes_{B} F$ we define the co-product $E \otimes_{B} F$ where $B=\{0,1\}$ (since + is idempotent).
%\end{definition}


%\begin{definition}
%The co-product $\mathbb{E} \sqcup \mathbb{F}$ of two c-semirings $\mathbb{E}$ and $\mathbb{F}$ is defined by 
%$\mathbb{B} \oplus \mathbb{E} \oplus \mathbb{F} \oplus \mathbb{E} \otimes_{\mathbb{B}} \mathbb{F}$
%$\mathbb{E} \otimes_{\mathbb{B}} \mathbb{F}$
%, where $ \mathbb{B}$ is the boolean c-semiring.
%modulo the equations $x + 1 = 1$ and $x \otimes 0 = 0$, for all x $\in \mathbb{E}\sqcup \mathbb{F}$.
%\end{definition}

\begin{definition}Let $\mathbb{E}$ and $\mathbb{F}$ be two c-semirings. Their composition written $\mathbb{E} \sqcup \mathbb{F} = \left\langle \mathbb{E} \sqcup \mathbb{F}, +, \times, \boldmath{1},\boldmath{0} \right\rangle$ is defined as :
	\begin{list}{-}{}
		\item The carrier is given by the co-product $\mathbb{E} \sqcup \mathbb{F}$, and we write $\lfloor \cdot \rfloor : \mathbb{E} \rightarrow \mathbb{E} \sqcup \mathbb{F}$ and $\lceil \cdot \rceil : \mathbb{F} \rightarrow \mathbb{E} \sqcup \mathbb{F}$ the left and right injections.
		\item The choice operator $+ : 2^{\mathbb{E} \sqcup \mathbb{F}} \rightarrow \mathbb{E} \sqcup \mathbb{F}$, where, given $e_1,e_2 \in \mathbb{E}$ and $s_1,s_2 \in \mathbb{F}$: : 
		\begin{align*}
			& \lceil e_1 \rceil + \lceil e_2 \rceil = \lceil e_1 + e_2 \rceil \\
			& \lfloor s_1 \rfloor + \lfloor s_2 \rfloor = \lfloor s_1 + s_2 \rfloor  %\\
			%			& \lfloor s_1 \rfloor \times \lceil e_1 \rceil + \lfloor s_2 \rfloor \times \lceil e_2 \rceil = \lfloor s_1 + s_2 \rfloor \times \lceil e_1 +e_2 \rceil
		\end{align*}
		\item The composition operator $\times : \mathbb{E} \sqcup \mathbb{F} \times \mathbb{E} \sqcup \mathbb{F} \rightarrow \mathbb{E} \sqcup \mathbb{F}$, where, given $e_1,e_2 \in \mathbb{E}$ and $s_1,s_2 \in \mathbb{F}$:
		\begin{align*}
			& \lceil e_1 \rceil \times \lceil e_2 \rceil = \lceil e_1 \times e_2 \rceil \\
			& \lfloor s_1 \rfloor \times \lfloor s_2 \rfloor = \lfloor s_1 \times s_2 \rfloor 
		\end{align*}
		\item $\textbf{0}_{\mathbb{E} \sqcup \mathbb{F}}=\lceil\textbf{0}_{\mathbb{E}}\rceil=\lfloor\textbf{0}_{\mathbb{F}}\rfloor$ and $\textbf{1}_{\mathbb{E} \sqcup \mathbb{F}}=\lceil\textbf{1}_{\mathbb{E}}\rceil=\lfloor\textbf{1}_{\mathbb{F}}\rfloor$
	\end{list}
\end{definition}
