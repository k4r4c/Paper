For this section, we assume $E=\langle E, \times,+,1,0 \rangle$ and $F=\langle F, \times,+,1,0 \rangle$ two c-semirings. 

\begin{definition} We define the disjoint union of two sets A and B while identifying elements of the intersection S:	
	$$A \cup_{S}B = (A \uplus B)/\sim, \quad x \sim y \Leftrightarrow x=y,\quad \text{where } y\in S$$
\end{definition}


Free(X) is the smallest set such that :
\begin{enumerate}
	\item $X \subseteq Free(X)$
	\item $x,y \in Free(X) \Rightarrow x \otimes y \in Free(X)$
	\item $x,y \in Free(X) \Rightarrow x + y \in Free(X)$
\end{enumerate}

We define ${\equiv} \subset Free(X)^2$ as the smallest congruence with regard to $\otimes,+$:
\begin{enumerate}
	\item $x \otimes y \equiv y \otimes x$
	\item $(x+x')\otimes y \equiv x \otimes y + x' \otimes y$
	\item $x+x \equiv x$
	\item $(x \otimes y) \otimes z \equiv x \otimes (y \otimes z)$
\end{enumerate}


\begin{definition} Given E,F csemiring and $\mathbb{B}$ the boolean csemiring, the tensor product  $E\otimes_{\mathbb{B}}F$ defined by the tuple $\langle Free(E \cup_{\mathbb{B}} F) /\equiv, +, \otimes, 0_{\mathbb{B}}, 1_{\mathbb{B}} \rangle $ is the co-product of E and F, where $0_{\mathbb{B}} \otimes a = 0_{\mathbb{B}}$, $1_{\mathbb{B}} + a = 1_{\mathbb{B}}$, $0_{\mathbb{B}} + a = a$ and $1_{\mathbb{B}} \otimes a = a$ for all $a \in E\otimes_{\mathbb{B}}F$
\end{definition}


%Since + is idempotent in a csemiring, the tensor product needs only be defined over a Boolean semiring. There exists a natural embedding of csemiring E and F into the tensor product $E \otimes_{\mathbb{B}} F$ with the maps $i_E : E \rightarrow E \otimes_{\mathbb{B}} F, e \longmapsto e $ and $i_F : F \rightarrow E \otimes_{\mathbb{B}} F, f \longmapsto f  $.
%\begin{figure}[H]
%	\centering
%	\resizebox{8cm}{!}{ \begin{tikzpicture}[>=latex,shorten >=1pt,node distance=3cm,on grid,auto, node/.style={circle,draw,minimum size=25pt}, ]
 
 \node[draw=none] (q0) at (0pt,0pt) {$G$};
 \node[draw=none, right = of q0] (q1) {$F$};
 \node[draw=none, left = of q0] (q2) {$E$};
 \node[draw=none, above = of q0] (q3) {$E \otimes_{\mathbb{B}} F$};
 \draw[->] (q1) to node[above right] {$i_F$} (q3);
 \draw[->] (q2) to node[above left] {$i_E$} (q3);
 \draw[->] (q1) to node[above] {f} (q0);
 \draw[->] (q2) to node[above] {e} (q0);
 \draw[dashed,->] (q3) to node[right] {g} (q0);
 \end{tikzpicture}
}
%	\caption{Coproduct universal property}
%	\label{coprod}
%\end{figure}