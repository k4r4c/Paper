%The need for preferences can be justify by some simple example. If we look at the example of Fig. 1, on the state $q_W$, three different transitions are allowed : from $q_W$ to $q_E$ with constraint $east$, from $q_W$ to $q_W$ with constraint $west$ and from $q_W$ to $q_W$ with constraint $stay_{lon}$. If all of those three constraints are satisfiable, we will pick one non deterministically. But what if we want, in this case, to determine which one we prefer ? How could we encode that we prefer taking a transition instead of another one ?
Semirings constitutes a suitable mathematical structure to define the notion of preferences. A constraint semiring \cite{B04} induces an order relation among its element. Therefore, following the initial example on stream and behavior, each action of the stream is attached a preference in a c-semiring. The csemiring will serve as algebraic structure to order streams. We will look at the composition of set of streams with preferences, and the resulting stream order. 
Given two systems $A_1$ and $A_2$ with ordered stream and synchronization constraints ($A_1$ must synchronize along its run with $A_2$), what is the ordered stream generated by the parallel composition of $A_1$ and $A_2$ ? To answer this question, we must study the properties of the csemiring and order induced by the composition of different csemirings. 

A csemiring has an induced order defined given by the properties of its two operators : $+$ and $\times$. The $+$ operator can intuitively be seen as choosing the best value between two value. Thus, if $a,b \in E$, $a+b=a$ can be interpreted as $a$ "is preferred" to $b$. Alternatively, $\times$ is intuitively used to compose preferences. Then $a \times b$ is a new preference, which could be different from $a$ or $b$.

%As we want to compose ordered stream, we 
Since csemiring define an order among its element, and we have "intuition" behind its operators, we now want to look at the composition of different csemiring, and get intuition behind the order defined on the composed csemiring. Recall that we also want the composition to be ordered, thus being described by a csemiring.

On the search of a structure for our product of csemiring that :
\begin{list}{-}{}
	\item composed preferences of composite actions.
	\item induces an order on the composition.
	\item can later on be composed with a new csemiring.
\end{list}

The carrier of the semiring should contain any possible elements from the composite csemiring. Then, we call the free product such structure, being the cartesian product of all csemiring involved in the composition. It is possible to show that the direct product of two csemiring is itself a csemiring, where $+$ and $\times$ operator are interpreted as $+$ and $\times$ of the underlying csemiring. Thus, if $A_1$ compose with $A_2$, where preferences of $A_1$ are defined over a csemiring $E_1 \times E_2$ and preferences of $A_2$ are defined over $E_3$, we get $A = A_1 \times A_2$ with preferences defined over $E_1 \times E_2 \times E_3$.

At this point, remark that the definition of the composition is not without implication. Recall the csemiring induces an order among its element. By defining the product of csemiring as the cartesian product with interpretation of $\times$ and $+$ as underlying csemiring operators, we also define a certain order among the generated streams of the composed system. In this case, the induced order is the following : chose the best preference of the left csemiring, and chose the best preference of the right csemiring, and the resulting preference is the tuple consisting of those two preferences. 

\begin{example}
	The case where $\langle west, east \rangle ^{\omega}$ is a possible behavior of the system, and are attached some preferences : $\langle (west ,2), (east,5) \rangle ^{\omega}$ where $2$ and $5$ are value of the weighted csemiring. The stream described by $\langle (stay_{lon} ,3) \rangle ^{\omega}$ is also a possible behavior, but since $2<3$, the action $west$ is preferred to the action $stay_{lon}$, therefore the behavior described by the stream $\langle (west ,2), (east,5) \rangle ^{\omega}$ is preferred to the behavior described by the stream $\langle (stay_{lon} ,3) \rangle ^{\omega}$. However, taking one step further, in the next element of the stream, we now compare $\langle (east,5),(west ,2) \rangle ^{\omega}$ with $\langle (stay_{lon} ,3) \rangle ^{\omega}$. Since $3<5$, the stream order changed, and the first behavior (which was $\langle (east,5),(west ,2) \rangle ^{\omega}$) is no more preferred to the other behavior  $\langle (stay_{lon} ,3) \rangle ^{\omega}$. The system will now prefer to take the action $stay_{lon}$. Repeating this process, we finally get the stream $\langle (west,2),(stay_{lon},3) \rangle ^{\omega}$ as a preferred behavior.
\end{example}

\begin{example}
	Suppose a composition with a new behavior $\langle (snap,0.5) \rangle ^{\omega}$. By definition, the composed behavior are $\langle ((west,snap),(2,0.5)),((east,snap),(5,0.5)) \rangle ^{\omega}$ and $\langle ((stay_{lon},\emptyset),(3,\emptyset)) \rangle ^{\omega}$.
\end{example}


