Specifying a system directly as a single Constraint Automata is not easy, since the number of states and transitions can become quite large. Instead, we want to see our system as a set of Constraint Automata in composition. We next define a composition operator for CA.

\begin{definition}
	Given $A_1=\left\langle Q_1, \rightarrow_1, C_1, q_{01}\right\rangle$ and $A_2=\left\langle Q_2, \rightarrow_2, C_2, q_{02} \right\rangle$ two constraints automata, we define the product $ A_1 \times A_2 = \left \langle Q, \rightarrow, C, (q_{01},q_{02}) \right \rangle$ where: 
	\begin{list}{-}{ }
		\item $Q= Q_1 \times Q_2 $ is a set of state and $(q_{01},q_{02})\in$ Q is the initial state.
		\item The transition relation $\rightarrow$ is the smallest relation satisfying
		$$
		\frac{q_{01} \xrightarrow{c_1}_1 q_{01}' ,\quad q_{02} \xrightarrow{c_2}_2 q_{02}'}{(q_{01},q_{02}) \xrightarrow{c_1 \land c_2}(q_{01}',q_{02}')}
		$$
		%A transition is denoted by $\left\langle q_i, c, s, p_i \right\rangle \in \rightarrow$
	\end{list}
\end{definition}

%The behavior of the composed automaton $\mathcal{A}= \mathcal{A}_1 \times \mathcal{A}_2$ can be seen as follow. 
The assignment map $\Gamma$ of $\mathcal{A}$ is the union of the assignment map $\Gamma_1$ of $\mathcal{A}_1$ and $\Gamma_2$ of $\mathcal{A}_2$. Which reflect the interaction between free variables shared between $\mathcal{A}_1$ and $\mathcal{A}_1$. 
Another illuminating view of the behavior of Constraint Automata in composition is given by thinking of concurrent executions. Both composite automaton tries to satisfy concurrently and consistently a constraint. In the case they succeed, they both take their transition, update the assignment map $\Gamma$ and perform this new concurrent execution on the new state.
It is possible to encode independent progress on each state $q\in Q$ on $\mathcal{A}$ by adding a transition $(q,\phi,q)$ where $\phi$ represents the constraint where no free variables of the automaton are defined $\bigwedge_{v \in V} (v\not=*)$.