
\begin{definition}
	A Constraint Automaton is a tuple $\left\langle Q, \rightarrow, C, q_{0}, \right\rangle$ where: 
	\begin{list}{-}{ }
		\item Q is a set of state and $q_0\in$ Q is the initial state.
%		\item C is a set of constraints, where a constraint is a predicate $c: (V_c \rightarrow D) \rightarrow \{\top,\bot\}$.
		\item C is a set of constraints.
		\item ${\rightarrow} \subseteq Q \times C \times Q$ is a finite relation called transition relation.
	\end{list}
		The tuple $(q,c,q') \in \rightarrow$ represents a transition from state q to state q' subject to the constraint c. We write $q \xrightarrow{c} q'$ instead of $(q,c,q') \in \rightarrow$. If c $\equiv \bot$, we can drop $q \xrightarrow{c} q'$ from $\rightarrow$.
\end{definition}

At this point, we should explain the behavior of a Constraint Automata $\mathcal{A}$. Before being able to do so, we must introduce a new notion. We call $\Gamma : V \rightarrow D\cup\{*\}$ the assignment map defined on the set of all free variables of $\mathcal{A}$. Given $v \in V$, $\Gamma(v)$ is either the symbol $*$ or a value from the domain $D$. We say that $v$ is undefined if $\Gamma(v)=*$. Otherwise, $\Gamma(v) \in D$ is a value of the free variable v.
Note that the context carried by $\Gamma$ can be modified either by a local assignment (a constraint of the shape $v=d$ where $d\in D$).
%Note that there is two kinds of variables $v$ : local or externally, by the environment (explained later, after defining composition). If, $\Gamma(v)=*$ and the taken transition contains an equality $v=d$ where $d \in D $, then the map $\Gamma$ is updated such that $\Gamma(v)=d$ (the value d is assigned to v). 
%Since variables could be shared between constraints, updating a variable in a transition the environment assigns a value to the variable. either the constraint on a transition contains an equality $v=d$ where $d \in D$, or the environment on a state fixes a value for $v$ (for instance, reading acquisition from a sensor). We define as interface the set of variable modifiable by the environment.

It is now possible to express the behavior of the automaton in terms of sequence of constraints. Starting in the initial state $q_0$ with an assignment map $\Gamma$, all outgoing transition are evaluated within $\Gamma$ : for all free variable $v$ such that $\Gamma(v) \not = *$, we substitute the value $\Gamma(v)$ for $v$ in the constraint. We then get a set of outgoing transitions whose constraints are satisfiable by $\Gamma$. We choose one constraint non deterministically, and update the assignment map $\Gamma$. We repeat this procedure in the next states.


