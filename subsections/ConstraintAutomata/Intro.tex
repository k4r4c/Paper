%Given the current computer architecture, and computational model, defining a system as a set of variables is adequate. Whether we look down to the transistor of a computer, or into a software program, we always face the same question : how to automatically go from a variable assignment to another variable assignment ?
%Since we want our model to restrict the possible behavior
Constraint Automata (CA) were introduced in \cite{MSA03} and define a compositional framework for the design of concurrent systems. CA are state transition system where transitions are labeled with constraints. We first present the language that defines the constraints, and then define the syntax and semantic of Constraint Automata. 
%From an operational perspective, each constraints represent a boolean condition to go from a source state to a target state. Constraints are first order logical formula in the language of constraint. 